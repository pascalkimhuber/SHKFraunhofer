% Created 2013-12-19 Thu 15:03
\documentclass[11pt]{article}
\usepackage[utf8]{inputenc}
\usepackage[T1]{fontenc}
\usepackage{graphicx}
\usepackage{longtable}
\usepackage{float}
\usepackage{wrapfig}
\usepackage{soul}
\usepackage{amssymb}
\usepackage{hyperref}


\title{todos}
\author{Pascal Huber}
\date{19 December 2013}

\begin{document}

\maketitle

\setcounter{tocdepth}{3}
\tableofcontents
\vspace*{1cm}
\section{Configurations}
\label{sec-1}


\subsection{\textbf{TODO} Change passwords}
\label{sec-1.1}

Ask Christian how to do this and which passwords have to be changed. 

\subsection{\textbf{TODO} Colors in git-output}
\label{sec-1.2}




\section{Emacs}
\label{sec-2}


\subsection{Repair yasnippet config <2013-12-16 Mon>}
\label{sec-2.1}


\subsection{Repair auto-pair config <2013-12-16 Mon>}
\label{sec-2.2}


\subsection{Language evaluation in org mode (see \href{http://zeekat.nl/articles/making-emacs-work-for-me.html#sec-10}{http://zeekat.nl/articles/making-emacs-work-for-me.html\#sec-10})}
\label{sec-2.3}




\section{Arbeitszeiten}
\label{sec-3}

\begin{itemize}
\item \textit{2013-12-16 Sat 16:30}--\textit{2013-12-16 Mon 18:00}
\item \textit{2013-12-18 Wed 17:30}--\textit{2013-12-18 Wed 19:56}
\end{itemize}
\section{Aufgaben}
\label{sec-4}


\subsection{\textbf{TODO} Go through tutorials and manual (find mistakes)}
\label{sec-4.1}


\subsubsection{\textbf{TODO} Bemerkungen}
\label{sec-4.1.1}

\begin{description}
\item vielleicht eine Hinweis, wie man an das Manual kommt?
\item [page 7] Letzter Satz: missing ``of'' after ``\ldots{} detailed specification''
\item [page 11] Erster Satz: es heisst \texttt{examples/Argon/} nicht \texttt{example/Argon/}.
\item [page 17] Verstehe nicht den Satz in der Klammer im Abschnitt \emph{parallelepiped}
\item [page 105] Dritter Satz fehlendes ``s'' in ``exist''
\item [page 105]Erhalte Warnung bei Aufruf von tremolo in Tutorial 14.1: \texttt{WARNING: Definition of unit system in the tremolo file is deprecated and may not be supported in future versions of tremolo. Define the unit system in the .potentials file instead.}
\item [page 107] oben, es fehlt ``use a'' in1 \emph{Since that introduces a discontinuity, we also second potential, which \ldots{}}
\end{description}
\subsubsection{\textbf{TODO} Notizen}
\label{sec-4.1.2}

\begin{itemize}

\item Overview\\
\label{sec-4.1.2.1}

Das Tremolo Projekt besteht aus zwei Funktionalitaeten:
a. sequentielle Simulation 
b. parallele Simulation

\begin{enumerate}
\item Different capabilities of tremolo

\begin{itemize}
\item ensembles
\item integrators
\item thermostats
\end{itemize}

\item In-depth description of the different potentials
\item Detailed specification of the syntax of options in the parameter file.
\end{enumerate}

\item First steps\\
\label{sec-4.1.2.2}

Start a tremolo simulation. Example:
Go to \texttt{tremole/examples/Argon/} and type \texttt{tremolo argon.tremolo}. This starts the simulation specified in the \texttt{*.tremolo} file. 


\item Ensembles and Thermostats\\
\label{sec-4.1.2.3}

Es gibt unter anderem folgende Funktionalitaeten:
\begin{enumerate}
\item Propagators
\item Thermostats
\item Barostats
\end{enumerate}

\item \textbf{TODO} Tutorial\\
\label{sec-4.1.2.4}

Kopie des Ordners \texttt{/tremolo/tutorial} befindet sich auf \texttt{\textasciitilde{}/playground/}. 

\begin{itemize}

\item \textbf{TODO} 14.1 Optimizing an initial particle setup\\
\label{sec-4.1.2.4.1}

\begin{enumerate}
\item Write a \texttt{*.tremolo}-file containing:

\begin{itemize}
\item defaultpath (has to be set!)
\item projectname (all files will carry this name)
\item comment
\item systemofunits
\item base magnitudes for the system of units used.
\end{itemize}

\item Write a \texttt{*.potentials}-file containing the potentials.

\begin{enumerate}
\item particles : Contains all particles in the simulation:

\begin{itemize}
\item particle$_{\mathrm{type}}$
\item element$_{\mathrm{name}}$
\item mass
\item sigma, sigma14, epsilon, epsilo14
\end{itemize}

\item potentials to be used in between particles
\end{enumerate}

\end{enumerate}
\end{itemize} % ends low level
\end{itemize} % ends low level
\subsubsection{\textbf{TODO} Fragen}
\label{sec-4.1.3}


\begin{enumerate}
\item $\Box$ Was sind Ensembles?
\end{enumerate}

\end{document}