\documentclass{scrartcl}

\usepackage{mathtools}
\usepackage{amssymb}

\begin{document}

\section{The basics}
\label{sec:basics}

\begin{itemize}
\item Hamilton's equation of motion:
\[ \dot{x} = \nabla_p \mathcal{H}(x,p) \hspace{1cm} \dot{p} = \nabla_x \mathcal{H}(x,p) \]
\item Hamiltonian \(\mathcal{H}\) with conservative potential \(U\)
  (\(U=U(x), \partial_t U = 0\)): 
\[\mathcal{H}(x,p) = \frac{1}{2} \sum_{i=1}^N \frac{p_i^T p_i}{m_i} + U(x_1, \dots, x_N) = E_{kin} + E_{pot} \]
\item \(\frac{d}{dt} \mathcal{H}(x,p) = 0\) microcanonical ensemble. 
\end{itemize}



\section{The ensembles}
\label{sec:ensembles}

A statistical ensemble is an idealization consisting of a large number of virtual copies (sometimes infinitely many) of a system, considered all at once, each of which represents a possible state that the real system might be in. In other words, a statistical ensemble is a probability distribution for the state of the system.

\subsection{The microcanonical ensemble (NVE)}
\label{sec:micr-ensemble-nve}

Statistical ensemble that is used to represent the possible states of
a mechanical system which has an exactly specified total energy. 
\begin{itemize}
\item isolated system: energy remains constant
\item Number, Volume, Energy are constant
\item the microcanonical ensemble is defined by assigning an equal
  probability to every microstate whose energy falls within a range
  centered at E. 
\end{itemize}

\subsection{The canonical ensemble (NVT)}
\label{sec:canon-ensemble-nvt}

Statistical ensemble that is used to represent the possible states of
a mechanical system which is in thermal equilibrium with a heat bath. 
\begin{itemize}
\item energy can vary
\item Number, Volume, Temperature remain constant
\item the canonical ensemble assigns a probability P to each
  microstate given by the following exponential: \(P =
  \exp(\frac{A-E}{kT})\). 
\end{itemize}

\subsection{The isothermal-isobaric ensemble (NPT)}
\label{sec:isoth-isob-ensemble}

Statistical ensemble that is used to represent the possible states of a
mechanical system which maintains constant temperature and constant
pressure. \\

\paragraph{Formulation of the isothermal-isobaric ensemble}
In order to get a formulation for the (NPT) ensemble the following
steps are carried out: 
\begin{itemize}
\item Consider the whole system together with the heat bath and the
  external piston as a NVE ensemble.\\
  This leads to 10 additional degrees of freedom (9 for the box-matrix
  entries and one for time scaling) by defining fictious potentials
  and additional dynamics: 
\item Use virtual variables for space and time (\(\tilde{h}_{i,j}, \gamma\)) and define
  fictious potentials (\(U_P, U_T\)), the so called Parrinello-Rahman
  barostat and the Nose thermostat. This gives the additional
  degrees of freedom.
\item By this one can define the Parrinello-Rahman-Nose Hamiltonian.
\item Transform back to real time (to get equidistant timesteps). 
  By this transformation the system becomes a non-Hamiltonian system. 
\end{itemize}

\subsection{The grand canonical ensemble (\(\mu\)VT)}
\label{sec:grand-canon-ensemble}

Statistical ensemble that is used to represent the possible states of a
mechanical system of particles that is being maintained in
thermodynamic equilibrium (thermal and chemical) with a reservoir. 
\begin{itemize}
\item can exchange energy and particles
\item chemical potential \(\mu\), Volume, Temperature are constant.
\end{itemize}


\section{Stress and strain}

\subsection{Stress \(\sigma = \frac{|\vec{F}|}{A}\)}

\begin{itemize}
\item Physical quantity expressing the internal forces that
  neighbouring particles of a continuous material exert on each other.
\item Stress is defined as the average force per unit area that some
  particle of a body exerts on an adjacent particle. (It is a
  macroscopic concept.)
\item Any strain (deformation) of a solid material generates an
  internal elastic stress. The relation between mechanical stress, deformation and the rate
  of change of deformation can be quite complicated.
\item The stress state of the material must be described by a tensor
  (Cauchy stress tensor \(\sigma\)). The stress tensor defines the
  state of stress at a point inside a material in the deformed
  placement or configuration. One has the relation \(\vec{T} =
  \sigma\cdot \vec{n}\), where \(\vec{T}\) is the stress vector for
  the plane described by the normal \(\vec{n}\). 
\end{itemize}

\subsection{Strain \(\sigma_{ij} = \sum_{k,l} C_{ijkl}
  \epsilon_{kl}\)}

\begin{itemize}
\item The strain describes the transformation of a body from a
  reference configuration to a current configuration. It describes the
  deformation in terms of relative displacement of particles in the
  body.
\item In a continuous material a deformation field results from a
  stress field. The relation between stresses and induced strains is
  expressed by constitutive equations (e.g. Hooke's law).
\item The strain tensor \(\epsilon\) describes what?
\end{itemize}

\subsection{Stress - Strain relation}

\begin{itemize}
\item Reference state: use matrix \(h_0\) (used to transform virtuel
  coordinates in real coordinates). A homogeneous distortion is then
  given by \(h_0 \leadsto h\).
\item This creates a displacement \(u\) which again yields in a
  formula for the strain tensor \(\epsilon\).
\item The coupling of strain and stress is given by ``Hooke's law'':
  \(\sigma_{ij} = \sum_{kl} C_{ijkl} \epsilon_{kl}\). 
\end{itemize}


\end{document}
