\documentclass{scrartcl}

\usepackage[utf8]{inputenc}
\usepackage{listings}
\usepackage{color}

\definecolor{mygreen}{rgb}{0,0.6,0}
\definecolor{mygray}{rgb}{0.5,0.5,0.5}
\definecolor{mymauve}{rgb}{0.58,0,0.82}

\lstset{ %
  backgroundcolor=\color{white},   % choose the background color; you must add \usepackage{color} or \usepackage{xcolor}
  basicstyle=\footnotesize,        % the size of the fonts that are used for the code
  breakatwhitespace=false,         % sets if automatic breaks should only happen at whitespace
  breaklines=true,                 % sets automatic line breaking
  captionpos=b,                    % sets the caption-position to bottom
  commentstyle=\color{mygreen},    % comment style
  deletekeywords={...},            % if you want to delete keywords from the given language
  escapeinside={\%*}{*)},          % if you want to add LaTeX within your code
  extendedchars=true,              % lets you use non-ASCII characters; for 8-bits encodings only, does not work with UTF-8
  frame=single,                    % adds a frame around the code
  keepspaces=true,                 % keeps spaces in text, useful for keeping indentation of code (possibly needs columns=flexible)
  keywordstyle=\color{blue},       % keyword style
  language=Octave,                 % the language of the code
  morekeywords={*,...},            % if you want to add more keywords to the set
  numbers=left,                    % where to put the line-numbers; possible values are (none, left, right)
  numbersep=5pt,                   % how far the line-numbers are from the code
  numberstyle=\tiny\color{mygray}, % the style that is used for the line-numbers
  rulecolor=\color{black},         % if not set, the frame-color may be changed on line-breaks within not-black text (e.g. comments (green here))
  showspaces=false,                % show spaces everywhere adding particular underscores; it overrides 'showstringspaces'
  showstringspaces=false,          % underline spaces within strings only
  showtabs=false,                  % show tabs within strings adding particular underscores
  stepnumber=2,                    % the step between two line-numbers. If it's 1, each line will be numbered
  stringstyle=\color{mymauve},     % string literal style
  tabsize=2,                       % sets default tabsize to 2 spaces
  title=\lstname                   % show the filename of files included with \lstinputlisting; also try caption instead of title
}

\begin{document}
\section{Writing Makefiles}
\label{sec:writing-makefiles}

\subsection{Targets, Rules, Dependencies}
\label{sec:targ-rules-depend}

Ein Makefile dient dazu, dem Programm \emph{make} mitzuteilen
\begin{itemize}
\item was es tun soll: \emph{Targets}
\item wie es das tun soll: \emph{Rules}
\end{itemize}
Werden für ein \emph{Target} andere Dateien benötigt, sogenannte
\emph{Dependencies}, müssen diese auch angegeben werden. \\

\paragraph{Bemerkung}
\label{sec:bemerkung}

\emph{Dependencies} sind für make auch \emph{Targets}, die an anderer
Stelle erstellt werden müssen.

\subsubsection{Beispiel}
\label{sec:beispiel}

Kompilierung eines einfachen C-Programms names prog aus den Dateien
prog.c und prog.h.

\lstset{language=make, caption={Einfaches Makefile mit Target
    ``prog''.}}
\begin{lstlisting}[frame=single]
  prog: prog.c prog.h
      gcc -o prog prog.c
\end{lstlisting}

\subsection{Das Defaulttarget}
\label{sec:das-defaulttarget}

Beim Aufruf von \emph{make} kann das zu erzeugende Target direkt
angegeben werden. Wenn nicht, wird das \emph{Defaulttarget}
erzeugt. Dies ist das erste Target im Makefile.

\subsubsection{Beispiel}
\label{sec:beispiel-1}

Kompilierung eines einfachen C-Programms mit anschließendem Aufruf von
\emph{make}.

\lstset{language=make, caption={Ein einfaches Makefile. Bemerke, dass
    in diesem Fall das Defaulttarget durch ``main'' gegeben ist.}}
\begin{lstlisting}[frame=single]
  # Simple Makefile for c-compiling.
  main: prog.o main.o
      gcc -o main prog.o main.o

  prog.o: prog.c
      gcc -c prog.c

  main.o: main.c
      gcc -c main.c

  clean:
      rm -rf *.o main
\end{lstlisting}

\lstset{language=bash, caption={Aufruf von make. In der ersten Zeile
    wird nur das Target main.o erzeugt. Im zweiten dann das
    Defaulttarget main.}}
\begin{lstlisting}
  $ make main.o
  $ make
\end{lstlisting}

\subsection{Pattern in Regeln}
\label{sec:pattern-regeln}

Es besteht die Möglichkeit Regeln durch eine Art
\emph{Wildcard-Pattern} zu defenieren. Wichtige vordefinierte
Variablen sind:
\begin{itemize}
\item \texttt{\$<} die erste Abhängigkeit
\item \texttt{\$@} Name des Targets
\item \texttt{\$+} eine Liste aller Abhängigkeiten
\item \texttt{\$\^} eine Liste aller Abhängigkeiten, ohne doppelte Einträge
\end{itemize}

\subsubsection{Beispiel}
\label{sec:beispiel-2}

\lstset{language=make, caption={Nutzung von Patterns zur Erzeugung von
  Objektdateien aus ihren jeweiligen c-Dateien.}}
\begin{lstlisting}[frame=single]
  %.o: %.c
      gcc -Wall -g -c $<
  main: prog.o main.o
      gcc -o main $^
\end{lstlisting}

\subsection{Variablen in Makefiles}
\label{sec:variablen-makefiles}
Es ist auch möglich in Makefiles \emph{Variablen} zu definieren. Dies
wird üblicherweise in Großbuchstaben gemacht. Beispielsweise:
\begin{description}
\item[\texttt{CC}] Compiler
\item[\texttt{CFLAGS}] Compiler-Flags
\item[\texttt{LDFLAGS}] Linker-Optionen
\end{description}
Der Zugriff auf die jeweilige Variable erfolgt durch \texttt{\$( )}.

\subsubsection{Beispiel}
\label{sec:beispiel-3}

\lstset{language=make, caption={Verwendung der Variable \texttt{CC}
    zur Beschreibung des Compilers.}}
\begin{lstlisting}
  CC = gcc

  main: prog.o main.o
      $(CC) -o main $^
\end{lstlisting}

\subsection{Kommentare in Makefiles}
\label{sec:kommentare-makefiles}

Kommentare in Makefiles können durch das Voranstellen einer Raute
\texttt{\#} eingefügt werden.

\subsection{Phony Targets}
\label{sec:phony-targets}

Im Allgemeinen prüft \emph{make}, ob ein Target aktueller ist als alle
Dependencies von den es abhängt. Ist dies nicht der Fall, wird es neu
erzeugt. \\
Manche Targets sollen aber unabhängig von ihren Dependencies immer
erzeugt werden. Solche Targets nennt man \emph{Phony}s.

\subsubsection{Beispiel}
\label{sec:beispiel-4}

Ein klassisches Beispiel für ein \emph{Phony} ist das Target
\texttt{clean}. Dieses Target ist in der Regel keine Datei und sollte
deshalb im Allgemeinen immer erzeugt werden. Problematisch wird dies
allerdings, wenn eine Datei mit dem selben Namen \texttt{clean}
existiert, dass von keinen anderen Dateien abhängt. In diesem Fall ist
das Target \texttt{clean} immer aktueller als seine Dependencies und
wird also nie erzeugt. Deshalb deklariert man das Target
\texttt{clean} häufig als Phony.

\lstset{language=make, caption={Das Target \texttt{clean} wird häufig
    als Phony deklariert.}}
\begin{lstlisting}[frame=single]
  .PHONY: clean
  clean:
      rm -rf $(BIN) $(OBJ)
\end{lstlisting}

\subsection{Pattern Substitution}
\label{sec:pattern-substitution}

Wie bei Rules kann man auch Variablen mit Hilfe von \emph{Pattern}
erzeugen.

\subsubsection{Beispiel}
\label{sec:beispiel-5}

\lstset{language=make, caption={Einsatz von Pattern zur Deklaration
    von Variablen. Hier werden die Objektdateien genutzt um die
    Source-Dateien zu deklarieren.}}
\begin{lstlisting}[frame=single]
  OBJ= datei1.o datei2.o datei3.o
  _SRC= $(OBJ:%.o=%.c) datei4.c
  SRC=/file/$(SRC)
\end{lstlisting}

\subsection{Abhängigkeiten als Target (make dep)}
\label{sec:abhang-als-targ}

Werden Objekt-Dateien durch Pattern erzeugt, entfallen die
Abhängigkeiten von Header-Dateien. Um dieses Problem zu umgehen,
definiert man meist ein Target names \texttt{dep}.

\subsubsection{Beispiel}
\label{sec:beispiel-6}

\lstset{language=make, caption={Hier wird das Target \texttt{dep}
    definiert, um Header-Dateien bei der Kompilierung mit
    einzubeziehen.}}
\begin{lstlisting}[frame=single]
  SRC = datei1.c datei2.c datei3.c
  DEPENDFILE = .depend

  dep: $(SRC)
      gcc -MM $(SRC) > $(DEPENDFILE)

  -include $(DEPENDFILE)
\end{lstlisting}

\subsection{Rekursives Make}
\label{sec:rekursives-make}

Für große Projekte sind die Source-Dateien oft auf verschiedene
Verzeichnisse verteilt. Es besteht die Möglichkeit für jedes dieser
Unterverzeichnisse ein separates Makefile zu schreiben, das von einem
zentralen Makefile im Root-Ordner aufgerufen wird.

\end{document}

%%% Local Variables:
%%% mode: latex
%%% TeX-master: t
%%% End:
