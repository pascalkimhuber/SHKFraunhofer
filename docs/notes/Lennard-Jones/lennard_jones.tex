\documentclass{scrartcl}
\usepackage{mathtools}
\usepackage{amsmath}
\usepackage{amssymb}

% Commands ------------------------------------------------------------------
\newcommand{\boldp}{\boldsymbol{p}}
\newcommand{\boldq}{\boldsymbol{q}}

\begin{document}

\section{Hessians for the Lennard-Jones pair potential}
\label{sec:hess-lenn-jones}

Set 
\begin{equation}
  \label{eq:1}
  V(r) =  4 \varepsilon \left( (\frac{\sigma}{r})^{12} -
    (\frac{\sigma}{r})^{6} \right) = 4 \varepsilon \left( R^{12} -
    R^6 \right),  \qquad \text{ with } R = \frac{\sigma}{r},  
\end{equation}
and 
\begin{equation}
  \label{eq:2}
  r(\boldp, \boldq) = \lVert \boldq - \boldp \rVert =
  \sqrt{\sum_{i=1}^d r_i^2} \qquad \text{ with } r_i \coloneqq (q_i - p_i).
\end{equation}

\subsection{Partial derivatives of \(r\)}
\label{sec:part-deriv-r}

We have (\(r = r(\boldp, \boldq)\)):
\begin{align}
\partial_{p_i} r(\boldp, \boldq) &= - \frac{r_i}{r},  \\
\partial_{q_j} r(\boldp, \boldq) &=  \frac{r_j}{r},  \\
\partial_{q_i}\partial_{p_i} r(\boldp, \boldq) &= - \partial_{p_i}\partial_{p_i} r(\boldp, \boldq) =  \frac{r_i^2}{r^3} - \frac{1}{r},  \\
\partial_{q_j}\partial_{p_i} r(\boldp, \boldq) &= - \partial_{p_j}\partial_{p_i} r(\boldp, \boldq) =  \frac{r_ir_j}{r^3}, \\
\end{align}

\subsection{One dimensional derivatives of the Lennard-Jones
  potential}
\label{sec:one-dimens-deriv}

We have 
\begin{align}
  V'(r) &= \frac{24 \varepsilon}{r} R^6 \left(1 - 2 R^6\right) \\
  V''(r) &= \frac{24 \varepsilon}{r^2} R^6 \left( 26 R^6 - 7 \right).
\end{align}

\subsection{Lennard-Jones forces and Hessians}
\label{sec:lennard-jones-forces}

We have 
\begin{align}
  \partial_{p_i} V(r(\boldp, \boldq)) &= - \frac{24 \varepsilon}{r^2} R^6 \left(1 - 2 R^6\right)r_i \\
  \partial_{q_i}\partial_{p_i} V(r(\boldp, \boldq)) &= - \partial_{p_i}\partial_{p_i} V(r(\boldp, \boldq)) = \frac{24 \varepsilon}{r^4} R^6 \left(8 - 28 R^6\right)r_i^2 - \frac{24 \varepsilon}{r^2} R^6 \left(1 - 2 R^6\right) \\ 
  \partial_{q_j}\partial_{p_i} V(r(\boldp, \boldq)) &= - \partial_{p_j}\partial_{p_i} V(r(\boldp, \boldq)) = \frac{24 \varepsilon}{r^4} R^6 \left(8 - 28 R^6\right)r_ir_j
\end{align}

\end{document}
%%% Local Variables:
%%% mode: latex
%%% TeX-master: t
%%% End:
