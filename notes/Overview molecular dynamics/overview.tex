\documentclass{scrartcl}

\usepackage{mathtools}
\usepackage{amssymb}

\begin{document}

\section{The basics}
\label{sec:basics}

\begin{itemize}
\item Hamilton's equation of motion:
\[ \dot{x} = \nabla_p \mathcal{H}(x,p) \hspace{1cm} \dot{p} = \nabla_x \mathcal{H}(x,p) \]
\item Hamiltonian \(\mathcal{H}\) with conservative potential \(U\)
  (\(U=U(x), \partial_t U = 0\)): 
\[\mathcal{H}(x,p) = \frac{1}{2} \sum_{i=1}^N \frac{p_i^T p_i}{m_i} + U(x_1, \dots, x_N) = E_{kin} + E_{pot} \]
\item \(\frac{d}{dt} \mathcal{H}(x,p) = 0\) microcanonical ensemble. 
\end{itemize}



\section{The ensembles}
\label{sec:ensembles}

A statistical ensemble is an idealization consisting of a large number of virtual copies (sometimes infinitely many) of a system, considered all at once, each of which represents a possible state that the real system might be in. In other words, a statistical ensemble is a probability distribution for the state of the system.

\subsection{The microcanonical ensemble (NVE)}
\label{sec:micr-ensemble-nve}

Statistical ensemble that is used to represent the possible states of
a mechanical system which has an exactly specified total energy. 
\begin{itemize}
\item isolated system: energy remains constant
\item Number, Volume, Energy are constant
\item the microcanonical ensemble is defined by assigning an equal
  probability to every microstate whose energy falls within a range
  centered at E. 
\end{itemize}

\subsection{The canonical ensemble (NVT)}
\label{sec:canon-ensemble-nvt}

Statistical ensemble that is used to represent the possible states of
a mechanical system which is in thermal equilibrium with a heat bath. 
\begin{itemize}
\item energy can vary
\item Number, Volume, Temperature remain constant
\item the canonical ensemble assigns a probability P to each
  microstate given by the following exponential: \(P =
  \exp(\frac{A-E}{kT})\). 
\end{itemize}

\subsection{The isothermal-isobaric ensemble (NPT)}
\label{sec:isoth-isob-ensemble}

Statistical ensemble that is used to represent the possible states of a
mechanical system which maintains constant temperature and constant
pressure. 

\subsection{The grand canonical ensemble (\(\mu\)VT)}
\label{sec:grand-canon-ensemble}

Statistical ensemble that is used to represent the possible states of a
mechanical system of particles that is being maintained in
thermodynamic equilibrium (thermal and chemical) with a reservoir. 
\begin{itemize}
\item can exchange energy and particles
\item chemical potential \(\mu\), Volume, Temperature are constant.
\end{itemize}


\end{document}
