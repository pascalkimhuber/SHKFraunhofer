\chapter{Potentials}

\todo{translate potential explanations to english.}

\todo{Citations need to be collected!}



\section{From PDB to Simulation}
\todo{Is this section (PDB) sensible for the distributed version of Tremolo-X?}

\todo{For current parser?}

\subsection{Protein Data Bank}
The coordinates of proteins can be obtained from the Protein Data Bank (PDB)
\cite{pdb-online}. The data therein is mostly obtained from NMR-measurements
or X-ray crystallography, thus coordinates for hydrogen atoms are usually missing
since those have little mass and are hard to measure in Experiment. The
PDB-format is published in a standard \cite{pdb-standard-online}.
\subsection{Hydrogenbuilder}
Due to chemical descriptions of the standard residues of proteins and some empirical
bonding rules for atoms it is possible to generate coordinates for hydrogen atoms in
retrospect. Commercial products such as HyperChem \cite{hyperchem-online} or CHARMM
are available online. Using so called topology descriptions the relevant data can
also be generated by hand.
\todo{reference to other commercial/open?}
\subsection{Waterbath}
For protein simulations usually the behavior in a solution (of water) is of interest.
Thus water molecules are added in a periodic cuboid around the molecule under investigation.
This can be done by HyperChem as well (but a small script serves the same purpose).
\todo{document utilities separately?}
\todo{reference to other commercial/open?}

\subsection{CHARMM data: Potential- and Topologyfiles}
Now the PDB data (augmented with hydrogen atoms and water molecules) needs to be
converted and parsed. Within the molecular dynamics program
CHARMM two essential files, describing this conversion, exist. (Other Force Fields,
s.a. AMBER, MM+ (HyperChem), GROMOS etc. have similar descriptions.)
Descriptions: The particles are distinguished by atom type. This is a finer
differentiation then by chemical elements, so that empirical knowledge and modeling
(such as bonding structure, etc.) may be used as well.

In the potential file a set of Lennard-Jones $(\epsilon, \sigma)$ parameters is
kept for each particle type. For pairs of particle types values for harmonic bonds
$( k_b, r_0)$, for atom triple angle data $(k_\theta,
\theta_0)$, and for 4-tuple $(k_\phi, n, \delta)$ and $(k_\psi, \psi_0)$
for torsion potentials and improper torsion potentials are kept. (The CHARMM-potential field uses
a sum of torsion contributions with a variety different of $n$).\\
\\
Example \cite{charmm-files} (following ! you find a comment):
\begin{lstlisting}
BONDS
!V(bond) = Kb(b - b0)**2
!Kb: kcal/mole/A**2
!b0: A
!atom type Kb          b0
CT1  C     250.000     1.4900
CT1  CC    200.000     1.5220

ANGLES
!
!V(angle) = Ktheta(Theta - Theta0)**2
!Ktheta: kcal/mole/rad**2
!Theta0: degrees
!atom types     Ktheta    Theta0
H    NH1  C      34.000   123.0000
H    OH1  CA     65.000   108.0000
\end{lstlisting}


The topology file describes the structure of the individual residues.
The name of the residue and the atom are taken from the PDB file, together with
the topology file the partial charges are determined, which are required to
determine the potential parameters for a particle and particle type.
Neighbor and bonding information is present as well. The description
in internal coordinates may be used for the addition of hydrogen atoms
to the PDB data.\\
\bigbreak
Example: Description of the residue Glycin \cite{charmm-files}:
\begin{lstlisting}
RESI GLY          0.00
GROUP
ATOM N    NH1    -0.47  !     |
ATOM HN   H       0.31  !     N-H
ATOM CA   CT2    -0.02  !     |
ATOM HA1  HB      0.09  !     |
ATOM HA2  HB      0.09  ! HA1-CA-HA2
GROUP                   !     |
ATOM C    C       0.51  !     |
ATOM O    O      -0.51  !     C=O
                        !     |
BOND N HN  N  CA  C CA
BOND C +N  CA HA1 CA HA2
DOUBLE O  C
IMPR N -C  CA HN  C CA   +N O
DONOR HN N
ACCEPTOR O C
IC -C   CA   *N   HN    1.3475 122.8200  180.0000 115.6200  0.9992
IC -C   N    CA   C     1.3475 122.8200  180.0000 108.9400  1.4971
IC N    CA   C    +N    1.4553 108.9400  180.0000 117.6000  1.3479
IC +N   CA   *C   O     1.3479 117.6000  180.0000 120.8500  1.2289
IC CA   C    +N   +CA   1.4971 117.6000  180.0000 124.0800  1.4560
IC N    C    *CA  HA1   1.4553 108.9400  117.8600 108.0300  1.0814
IC N    C    *CA  HA2   1.4553 108.9400 -118.1200 107.9500  1.0817
PATCHING FIRS GLYP
\end{lstlisting}

An accurate description of the syntax can be found in the
CHARMM-documentation \cite{charmm27b-online}.

Furthermore, CHARMM permits additional terms for hydrogen bods, however those
are not strictly necessary --- hydrogen bonds are modeled by coulomb interaction.
(Assuming accurate partial charges.)

\subsection{Conversion and Parsing}
A (Perl-)script the reads the PDB-data, the potential parameters and the topology
information, coordinates atoms with their particle types and partial charges,
constructs a list of neighbors and produces a file with all necessary particle and
potential information.
Those files are then used by Tremolo-X.

\section{Lennard-Jones Potentials}

Tremolo-X supplies several implementations of Lennard-Jones potentials. In addition to the standard 12-6 potential, the user may set the exponents on his own.
Since the use of a cut-off radius may lead to discontinuities in the particle forces and energies, we supply two different spline interpolations, smooth the
potential to a continuously differentiable one.\\
Note that the Lennard-Jones parameters $\epsilon$ ($\epsilon_{14}$) and $\sigma$ ($\sigma_{14}$) are used as parameters for some other potentials as well.
The parameters for the potential between particles of different types are created using the Lorentz-Berthelot mixing methods:
\begin{align*}
\epsilon &= \sqrt{\epsilon_1*\epsilon_2}\\
  \sigma &= \frac{\sigma_1+\sigma_2}{2}
\end{align*}
\todo{SlaterKirkwood written but not implemented yet.}

\paragraph{Standard 12-6 Lennard-Jones}
\begin{align*}
U = 4 \cdot \epsilon \left[ \left( \frac{\sigma}{r_{ij}}\right)^{12}-\left(\frac{\sigma}{r_{ij}} \right)^6 \right]
\end{align*}
\begin{lstlisting}
 nonbonded_2body_potentials      {
        lennardjones:   particle_type1=H,
                        particle_type2=H,
                        r_cut=1;
        };
\end{lstlisting}
Particle types may be set to any particle you specified in the particle section of this file.

The {\tt r\_cut} denotes the radius, at which the potential is cut of.


\paragraph{M-N-Lennard-Jones}
\todo{Bug/Inconsistency in tremolo with units/power of A and B.}
\begin{align*}
U = 4 \cdot \epsilon \left[ \left( \frac{A_{ij}}{r_{ij}}\right)^m-\left(\frac{B_{ij}}{r_{ij}} \right)^n \right]
\end{align*}

\begin{lstlisting}
 nonbonded_2body_potentials      {
        m_n_lennardjones:       particle_type1=H,
                                particle_type2=H,
                                a=0.44117647,
                                b=0.44117647,
                                m=12,
                                n=6,
                                r_cut=1.6176471;
        };
\end{lstlisting}
Particle types may be set to any particle you specified in the particle section of this file.

The {\tt r\_cut} denotes the radius, at which the potential is cut of.


\subsection{Spline Interpolation}
The smoothing for the Lennard-Jones potential is implemented by the multiplication of a spline to the potential function: $U_{smooth}=U \cdot S_i$.
The following two spline interpolations are available in Tremolo-X:

\paragraph{Spline I} \todo{find first spline in code and verify}
\begin{align*}
S_{I}(r) = \left\{ \begin{array}{ll} 1 &:r \leq r_l \\1-\left( r -r_l\right)^2\left(3r_{cut}-r_l- 2r\right)/\left(r_{cut}-r_l\right)^3&:r_l < r < r_{cut} \\0&:r\geq r_{cut} \end{array} \right.
\end{align*}

\begin{lstlisting}
 nonbonded_2body_potentials      {
        ljspline:       particle_type1=H,
                        particle_type2=H,
                        r_cut=3,
                        r_l=2.5;
        };
\end{lstlisting}
Particle types may be set to any particle you specified in the particle section of this file.

The {\tt r\_cut} denotes the radius, at which the potential is cut of.

{\tt r\_l} is the radius, where spline interpolation from the actual potential to 0 (at {\tt r\_cut}) starts.


\paragraph{Spline II} \todo{interpret second spline parameters in code.}
\begin{align*}
S_{II}(r) = \left\{ \begin{array}{ll} 1 &:r \leq r_s \\ &:r_s < r < r_{m} \\  &:r_m < r < r_{e} \\  &:r_e < r < r_{b} \\ 0&:r\geq r_{b} \end{array} \right.
\end{align*}

\begin{lstlisting}
 nonbonded_2body_potentials      {
        ljspline2:      particle_type1=H,
                        particle_type2=H,
                        r_s=0.88,
                        r_m=1.029,
                        r_e=1.470,
                        r_b=1.617;
        };
\end{lstlisting}
Particle types may be set to any particle you specified in the particle section of this file.


\section{Non-bonded Potentials}

\subsection{Brenner}
\todo{Double check potential (taken from book):}
This potential is designed for the use with Carbon and Hydrogen atoms.
\begin{align*}
U =\sum_{i=1}^N \sum_{j=1,j>i}^N f_{ij}(r_{ij})\frac{c_{ij}}{s_{ij}-1}\left[ U_{R}(r_{ij})-\bar{B}_{ij}U_A(r_{ij})\right]
\end{align*}
with a repulsive ($U_R$) and an attractive ($U_A$) component:
\begin{align*}
U_R(r_{ij})&= \exp\left( -\sqrt{2s_{ij}}\beta_{ij}(r_{ij}-r_{ij,0})\right) \\
U_A(r_{ij})&= s_{ij}\cdot \exp\left( -\sqrt{\frac{2}{s_{ij}}}\beta_{ij}(r_{ij}-r_{ij,0})\right).
\end{align*}
Furthermore, we have:
\begin{align*}
f_{ij}(r) = \left\{ \begin{array}{ll} 1 & :r<r_{ij,1}\\\frac{1}{2}\left[ 1+\cos\left(\pi\frac{r-r_{ij,1}}{r_{ij,2}-r_{ij,1}} \right)\right]&:r_{ij,1}\leq r<r_{ij,2}  \\0&: r_{ij,2}\leq r \end{array}\right.
\end{align*}
and
\begin{equation*}
\bar{B}_{ij}=\frac{(B_{ij}+B_{ji})}{2} + K(N_i, N_j, N_{ij}^{conj}),
\end{equation*}
where
\begin{align*}
B_{ij} = \left( 1+H_{ij}(Ni^H,N_i^C)+\sum_{k=1, k\neq i,j}G_i(\theta_{ijk})f_{ik}(r_{ik}) \exp\left( \alpha_{ijk}(r_{ij}-R_{ij}-r_{ik}+R_{ik})\right)  \right)^{-\delta_i}.
\end{align*}
The index of $G_i()\theta_{ijk}$ is used to indicate the dependency on the type of the $i^{th}$ atom:
\begin{align*}
G_{H} &= 12.33\\
G_{C} &= a_0 \left( 1+\frac{c_0^2}{d_0^2} - \frac{c_0^2}{d_0^2 +(1+\cos(\theta_{ijk}))^2 }\right).
\end{align*}
Additionally we define:
\begin{align*}
N_i^C &= \sum_{j \in C} f_{ij}(r_{ij})\\
N_i^H &= \sum_{j \in H} f_{ij}(r_{ij})\\
N_{ij}^{conj} &= 1+\sum_{k \in C, k\neq i,j} f_{ik}(r_{ik})F(N_k-f_{ik}(r_{ik}))+\sum_{k \in C, k\neq i,j} f_{jk}(r_{jk})F(N_k-f_{jk}(r_{jk}))
\end{align*}
and
\begin{align*}
F(z)=\left\{ \begin{array}{ll} 1&:z \leq 2\\ \frac{1}{2} \left[ 1+\cos( \pi(z-2)) \right]&:2<z\leq 3\\ 0&:z\geq 3\end{array} \right. .
\end{align*}

The functions $H_{ij}$ and $K$ are spline functions, smoothing the change from bonded to non-bonded state and among neighbors, respectively.

Values fitting the fixed parameters can be found in the following table:
\todo{Complete table Brenner parameters.}
\begin{tabular}{ll}
$a_0$ = & 0.00020813 \\
$c_0$ = & 330\\
$d_0$ = & 3.5
\end{tabular}

\begin{lstlisting}
brenner {
        brenner:        particle_type1=H,
                        particle_type2=H,
                        brennerhydrogen=off,
                        parameterset=I,
                        brennerlj=off;
        };
\end{lstlisting}
\todo{document parametersets and brennerlj.}


\subsection{Coulomb-ERFC}
\todo{Double check potential (taken from tooltip):}
\begin{align*}
u(r_{ij}) = \frac{e^2}{4\pi \epsilon_0}\frac{q_i q_j}{r_{ij}} \text{erfc}\left(\frac{r_{ij}}{\beta_{ij}}\right)
\end{align*}

\begin{lstlisting}
 nonbonded_2body_potentials      {
        coulomb_erfc:   particle_type1=H,
                        particle_type2=H,
                        r_cut=3.2352941,
                        beta=0.68823529;
        };
\end{lstlisting}
Particle types may be set to any particle you specified in the particle section of this file. Note that the coulomb-force-constant \verb+epsilon0inv+ has to be specified, see also section \ref{sub:coulomb}.

\subsection{Coulomb-QTaper}
\todo{Double check potential (taken from tooltip):}

The QTaper is only applied if the Coulomb potebtial is used. It is
used to avoid a collaps of a positive and a negative charged
particle. In Tremolo-X it is implemented in the form:
\begin{align*}
  u(r_{ij}) = left(\frac{e^2}{4\pi \epsilon_0}\frac{q_i q_j}{r_{ij}} -C)\times (f(r_{ij})-1)
\end{align*}
where the taper is given by
\begin{align*}
  f(r) = \frac{e^2}{4\pi \epsilon_0}\frac{q_i q_j}{r_{ij}}.
\end{align*}
and $r_0$ is the cutoff.
The Coulomb potential together with the QTaper results in the short range $[0,r_0]$ in:
\begin{align*}
   \frac{e^2}{4\pi \epsilon_0}\frac{q_i q_j}{r_{ij}}f(r_{ij}) +C \times (1-f(r_{ij})).
\end{align*}
See also \cite{gastreich2003charged}.

The parameters can be specified by:
\begin{lstlisting}
  nonbonded_2body_potentials {
    qtaper:  particle_type1=N,       particle_type2=B,  r_0=2.35, c=-17.9986;
    qtaper:  particle_type1=N,       particle_type2=Si, r_0=2.6,  c=-20.4023;
  };
\end{lstlisting}
Particle types may be set to any particle you specified in the particle section of this file. Note that the coulomb-force-constant \verb+epsilon0inv+ has to be specified, see also section \ref{sub:coulomb}.

\subsection{RSL2}
%Checked
\begin{align*}
U = \frac{a_{ij}}{1+\exp\left(b_{ij}(r_{ij}-c_{ij})\right)}
\end{align*}

\begin{lstlisting}
 nonbonded_2body_potentials      {
        rsl2:   particle_type1=H,
                particle_type2=H,
                r_cut=1.6176471,
                a=98.384782,
                b=20.4, c=0.64705882;
        };
\end{lstlisting}
Particle types may be set to any particle you specified in the particle section of this file.

\subsection{Stilling-Weber}
The Stillinger-Weber potential model \cite{stillinger} implemented in Tremolo-X reads as:
\begin{align*}
  U = \sum_{i<j} u^{ij}_{2}(r_{ij}) + \sum_{i<j<k} \left(u^{ijk}_{3}(r_{ji}, r_{jk}, \theta_{ijk}) + u^{jik}_{3}(r_{ij}, r_{ik}, \theta_{jik}) + u^{ikj}_{3}(r_{ki}, r_{kj}, \theta_{ikj})\right),
\end{align*}
where $\theta_{ijk}$ is the angle between $\vec{r}_{ji}:=\vec{r}_i-\vec{r}_j$ and $\vec{r}_{jk}:=\vec{r}_k-\vec{r}_j$. 

\subsubsection{Stilling-Weber 2body}
\todo{Double check potential (taken from tooltip):}
The Stillinger-Weber two-body potential reads as:
\begin{align*}
u^{ij}_{2}(r) = A\cdot(B\cdot r^{-p}-1)\cdot \exp\left({\frac{\gamma}{r-r_{cut}}}\right)
\end{align*}
Note that all parameters $A$, $B$, $\gamma$ and $r_{cut}$ depend on the particle types of particle pair $(i,j)$ and are symetric, i.e.\ e.g.\ $A_{ij} = A_{ji}$.

The parameters can bespecified by:
\begin{lstlisting}
  nonbonded_2body_potentials      {
    stiwe:  particle_type1=Si,      
            particle_type2=Si,      
            r_cut=3.77118,  
            p=4,    
            A=15.2855528754191,     
            B=11.6031922833963,     
            gamma=2.0951;
          };
\end{lstlisting}
Particle types may be set to any particle you specified in the particle section of this file.

\subsubsection{Stilling-Weber 3body}
\todo{Double check potential (taken from tooltip):}
The three-body potential reads as:
\begin{align*}
u_3^{type1}(r_{ji}, r_{jk}, \theta)& = \left(\cos(\theta)-\cos(\theta_0)\right)^2\cdot\lambda\cdot\exp\left(\frac{\gamma_0}{r_{ji}-r_0}+\frac{\gamma_1}{r_{jk}-r_1}\right)\\
u_3^{type2}(r_{ji}, r_{jk}, \theta)& = \left(\cos(\theta)-\cos(\theta_0)\right)\cdot\sin(\theta)\cdot\cos(\theta)\cdot\lambda\cdot\exp\left(\frac{\gamma_0}{r_{ji}-r_0}+\frac{\gamma_1}{r_{jk}-r_1}\right)
\end{align*}
Note that the parameters $\gamma_0$, $\gamma_1$, $r_0$, $r_1$ depend
on the particle types of the particle triple $(i,j,k)$, where symmetry
$(i,j,k) = (k,j,i)$ is assumed.

The parameters can be specified by:
\begin{lstlisting}
stiwe3bodys     {
        stiwe3body: particle_type1=Si,      
                    particle_type2=Si,      
                    particle_type3=Si,      
                    r_0=3.77118,    
                    gamma_0=2.51412,        
                    lambda=45.5343, 
                    r_1=3.77118,    
                    costheta0=-0.333333333333,      
                    gamma_1=2.51412,        
                    type=1;
                  };
\end{lstlisting}
Particle types may be set to any particle you specified in the
particle section of this file, where \verb+particle_type2+ is the
center particle.

\subsubsection{Parameters}
Examples of parameter sets for the Stillinger-Weber potential model are given in Table \ref{tab:StillingerWeber}.

\begin{table}
  \caption{Different parameter sets for the Stillinger-Weber potenial model}
  \label{tab:SuttonChen}
  \begin{center}
    \begin{tabular}{ccc}
      Elements & References & \texttt{.potentials} file \\\hline
      Si  & \cite{stillinger} & \verb+stiwe-Si_1985+\\\hline
      Mo, S & \cite{jiang2013molecular} & \verb+stiwe-MoS_2013+\\\hline
    \end{tabular}
  \end{center}
\end{table}

\subsection{Sutton-Chen}
The so called Sutton-Chen potential for fcc metals was introduced in \cite{sutton1990long} and for metal alloys in \cite{rafii1991long}.
The energy is composed of
\begin{align*}
  U = \sum_i \left[\sum_{j> i} \left( \frac{\epsilon_{ij}\sigma_{ij}} {r_{ij}}\right)^{n_{ij}} -\epsilon_{i}c_i\sqrt{\rho_i}\right] , \qquad \rho_i=\sum_{j\neq i} \left(\frac{\sigma_{ij}} {r_{ij}}\right)^{m_{ij}},
\end{align*}
where $\epsilon_i = \epsilon_{ii}$, $m_i = m_{ii}$ and $n_i = n_{ii}$.
\subsubsection{User entries}
\begin{lstlisting}
 nonbonded_2body_potentials      {
        suttonchen:     particle_type1=Ag,
                        particle_type2=Ag,
                        epsilon=0,
                        sigma=1.0,
                        r_cut=0,
                        m=6,
                        n=12,
                        c=0;
        suttonchen:     particle_type1=Ag,
                        particle_type2=Au,
                        r_cut=0,
                        m=6,
                        n=12;
                      };
\end{lstlisting}

Particle types may be set to any particle you specified in the
particle section of this file.  Note that in the case of
\verb+particle_type1+==\verb+particle_type2+, the parameters
\verb+r_cut+, \verb+m+, \verb+n+, \verb+c+ are necessary to be
specified.  In the case of
\verb+particle_type1+!=\verb+particle_type2+, no further parameters
have to be specified, e.g.\
\begin{lstlisting}
        suttonchen:     particle_type1=Ag,
                        particle_type2=Au;
\end{lstlisting}
since a not given parameter is given by the
respective mixing rule:
\begin{align*} %\label{equ:SuttonChenMixingRules}
  \epsilon_{ij} = \sqrt{\epsilon_i\epsilon_j} , \,
  \sigma_{ij}  = \frac{\sigma_i+\sigma_j}{2} , \,\\
  m_{ij} = \frac{m_i+m_j}{2} , \,
  n_{ij} = \frac{n_i+n_j}{2} , \,
  r^{cut}_{ij} = \frac{r^{cut}_{i}+r^{cut}_{j}}{2} , \,
\end{align*}
Note that, if \verb+epsilon+ or \verb+sigma+ is not set within
\verb+sutttonchen+, then their values are taken from the particle
data. Note finally that the parameter \verb+c+ does not need to be
specified for pairs of atom types, since its unused in that case.

\subsubsection{Parameters}
Examples of parameter sets for the Sutton-Chen potential model are given in Table \ref{tab:SuttonChen}.

\begin{table}
  \caption{Different parameter sets for the Sutton-Chen potenial model}
  \label{tab:SuttonChen}
  {\small
    \begin{center}
      \begin{tabular}{ccc}
        Elements & References & \texttt{.potentials} file \\\hline
        Ni, Cu, Rh, Pd, Ag,  & \cite{rafii1991long} & \verb+suttonchen-original-1991+\\
         Ir, Pt, Au, Pb, Al &  & \\\hline
        Ni, Cu, Rh, Pd, Ag, Ir, Pt, Au & \cite{kimura1998quantum} & \verb+suttonchen-original-1998+\\
         &  & \verb+suttonchen-classical-1998+\\
         &  & \verb+suttonchen-quantum-1998+\\\hline
         Ni, Cu, Ag, Au, Pt, Rh & \cite{ccaugin1999thermal} & \verb+suttonchen-NiCuAgAuPtRh-1999+\\\hline
         Fe & \cite{belonoshko2000quasi} & \verb+suttonchen-Fe-2000+\\\hline
         Ni, Al & \cite{kazanc2008investigation} & \verb+suttonchen-NiAl-2008+\\\hline
      \end{tabular}
    \end{center}
  }
\end{table}

\subsection{EAM}
\label{potentials:eam}
The so called Embedded-Atom Method potential for metals was introduced in \cite{daw1983semiempirical,daw1984embedded}.
In the  so-called EAM/alloy variant the energy is composed of
\begin{align}\label{equ:EAMAlloy}
  U = \sum_{i<j} \phi_{ij}(r_{ij}) +\sum_i F_i\left(\sum_{j\neq i}\rho_j(r_{ij})\right)
\end{align}
where $F$ is the embedding function, $\rho$ is the electron density
and $\phi$ a pair potential. Note that these functions are usually
given by tabulated functions. There, exists a slightly modified
so-called EAM/FS variant which allows for pair dependent $\rho$ like
in the case of Finnis-Sinclair potential model
\cite{finnis1984simple}. The energy is given by
\begin{align}\label{equ:EAMFS}
  U = \sum_{i<j} \phi_{ij}(r_{ij}) +\sum_i F_i\left(\sum_{j\neq i}\rho_{ij}(r_{ij})\right)
\end{align}

\subsubsection{User entries}
For the EAM/alloy variant (\ref{equ:EAMAlloy}) a potential file in the
EAM/alloy setfl format (see
e.g. \url{http://www.ctcms.nist.gov/potentials}) has to be specified:
\begin{lstlisting}
  eam      {
    setfl:          file="Fe-Ni.eam.alloy";   
  };
\end{lstlisting}
For the EAM/FS variant (\ref{equ:EAMFS}) a potential file in the
EAM/alloy setfl format (see
e.g. \url{http://www.ctcms.nist.gov/potentials}) has to be specified:
\begin{lstlisting}
  eam      {
    fssetfl:          file="Fe-C_Hepburn_Ackland.eam.fs";   
  };
\end{lstlisting}

Particle types are given in the 4th line of the setfl potential file and may be  set to any particle you specified in the
particle section of this file.  

\subsubsection{Parameters}
Examples of parameter sets for the EAM potential model are given in
Table \ref{tab:EAM-A} and Table \ref{tab:EAM-B}. Note that these parameter sets were not created by 
Fraunhofer SCAI, but compiled from public available source for your
convenience. As with any potential set, it is the responsibility of the user to check whether a
given parameter set is suitable for the intended application. We list the
sources of all compiled parameter sets in the table, as well as in the respective 
potential file.

In particular, see also the web site \url{http://www.ctcms.nist.gov/potentials}.

\begin{table}
  \caption{Selection of different parameter sets for the EAM potenial model.}
  \label{tab:EAM-A}
  {\footnotesize
    \begin{center}
      \begin{tabular}{ccc}
        Elements & References & \texttt{.potentials} file\\\hline
        Ag & \cite{ackland1987simple} & \verb+eamfs-Ag-Ackland-1987+\\
        &\cite{williams2006embedded} & \verb+eamalloy-Ag-Williams-2006+\\
        & \cite{zhou2004misfit} & \verb+eamalloy-Ag-Zhou-2004+\\\hline
        Ag, Cu &\cite{williams2006embedded} & \verb+eamalloy-AgCu-Williams-2006+\\
        & \cite{wu2009cu} & \verb+eamalloy-AgCu-Wu-2009+\\\hline
        Al & \cite{mishin1999interatomic} & \verb+eamalloy-Al-Mishin-1999+\\
        & \cite{sturgeon2000adjusting} & \verb+eamfs-AlMDSL-Sturgeon-2000+\\
        & \cite{zope2003interatomic} & \verb+eamalloy-Al-Zope-2003+\\
        & \cite{liu2004aluminium} & \verb+eamalloy-Al-Liu-2004+\\
        & \cite{zhou2004misfit} & \verb+eamalloy-Al-Zhou-2004+\\
        & \cite{mendelev2008analysis} & \verb+eamfs-Al1-Mendelev-2008+\\
        & \cite{winey2009thermodynamic} & \verb+eamalloy-Al-Winey-2009+\\\hline
        Al, Fe & \cite{Mendelev2005} & \verb+eamfs-AlFe-Mendelev-2005+ \\\hline
        Al, H, Ni& \cite{angelo1995trapping} & \verb+eamalloy-AlHNi-Angelo-1995+\\\hline
        Al, Ni & \cite{mishin2002embedded} & \verb+eamalloy-AlNi-Mishin-2002+\\
        & \cite{mishin2004atomistic} & \verb+eamalloy-AlNi-Mishin-2004+\\
        & \cite{purja2009development} & \verb+eamalloy-AlNi-Mishin-2009+\\\hline
        Al, Mg & \cite{liu1997anisotropic} & \verb+eamalloy-AlMg-Liu-1997+\\
        & \cite{mendelev2009development} & \verb+eamfs-AlMG-Mendelev-2009+\\\hline
        Al, Mn, Pd & \cite{schopf2012embedded} & \verb+eamalloy-AlMnPd-Schopf-2012+\\\hline
        Al, Pb & \cite{landa2000development} & \verb+eamalloy-AlPb-Landa-2000+\\\hline
        Al, Ti & \cite{zope2003interatomic} & \verb+eamalloy-AlTi-Zope-2003+\\\hline
        Au & \cite{ackland1987simple} & \verb+eamfs-Au-Ackland-1987+\\
        & \cite{zhou2004misfit} & \verb+eamalloy-Au-Zhou-2004+\\
        & \cite{grochola2005fitting} & \verb+eamalloy-Au-Grochola-2005+\\\hline
        C, Fe  & \cite{hepburn2008metallic} & \verb+eamfs-CFe-Hepburn-2008+\\\hline
        Co & \cite{zhou2004misfit} & \verb+eamalloy-Co-Zhou-2004+\\
        & \cite{pun2012embedded} & \verb+eamalloy-Co-PujaPun-2012+\\\hline
        Cu & \cite{ackland1987simple} & \verb+eamfs-Cu-Ackland-1987+\\
        & \cite{mishin2001structural} & \verb+eamalloy-Cu-Mishin-2001+\\
        & \cite{zhou2004misfit} & \verb+eamalloy-Cu-Zhou-2004+\\
        & \cite{mendelev2008analysis} & \verb+eamfs-Cu-Mendelev-2008+\\\hline
        Cu, Fe, Ni & \cite{bonny2009ternary} & \verb+eamalloy-CuFeNi-Bonny-2009+\\\hline
        % Cu, Pb & \cite{hoyt2003embedded} & \verb+eamfs-CuPb-Hoyt-2003+\\
        % Note that the potential from Hoyt was tested in October 2013 by Christian Neuen
        % and showed strange behaviour with atoms being expelled from the test particle
        % and the slowing to a complete stop. As a result this potential is not included
        % in our database.
        Cu, Zr & \cite{mendelev2009developmentcuzr} & \verb+eamfs-CuZr-Mendelev-2009+\\
        & \cite{mendelev2007using} & \verb+eamfs-CuZr-Mendelev-2007+\\\hline
        Fe  & \cite{ackland1997computer} & \verb+eamfs-Fe-Ackland-1997+\\
        & \cite{mendelev2003development} & \verb+eamfs-Fe2-Mendelev-2003+\\
        & \cite{mendelev2003development} & \verb+eamfs-Fe5-Mendelev-2003+\\
        & \cite{zhou2004misfit} & \verb+eamalloy-Fe-Zhou-2004+\\\hline
        Fe, Ni  & \cite{bonny2009fe} & \verb+eamalloy-FeNi-Bonny-2009+\\
        & \cite{meyer1995molecular} & \verb+eamalloy-FeNi-MeyerEntel-1995+\\\hline
        Fe, P & \cite{ackland2004development} & \verb+eamfs-FeP-Ackland-2004+\\\hline
        Fe, V & \cite{mendelev2007simulation} & \verb+eamfs-FeV-Mendelev-2007+\\\hline
        H, Pd & \cite{zhou2008embedded} & \verb+eamalloy-HPd-Zhou-2007+\\\hline
        Mo & \cite{zhou2004misfit} & \verb+eamalloy-Mo-Zhou-2004+\\\hline
        Mo, U, Xe & \cite{smirnova2013ternary} & \verb+eamfs-MoUXe-Smirnova-2013+\\\hline
        Mg & \cite{zhou2004misfit} & \verb+eamalloy-Mg-Zhou-2004+\\
        & \cite{sun2006crystal} & \verb+eamfs-Mg-Sun-2006+\\\hline
        Nb & \cite{fellinger2010force} & \verb+eamalloy-Nb-Fellinger-2010+\\\hline
      \end{tabular}
    \end{center}
  }
\end{table}
\begin{table}
  \caption{Selection of different parameter sets for the EAM potenial model.}
  \label{tab:EAM-B}
  {\footnotesize
    \begin{center}
      \begin{tabular}{ccc}
        Elements & References & \texttt{.potentials} file\\\hline
        Ni & \cite{ackland1987simple} & \verb+eamfs-Ni-Ackland-1987+\\
        & \cite{mishin1999interatomic} & \verb+eamalloy-Ni-Mishin-2009+\\
        & \cite{zhou2004misfit} & \verb+eamalloy-Ni-Zhou-2004+\\\hline
        & \cite{mendelev2012development} & \verb+eamfs-Ni1-Mendelev-2012+\\\hline        
        Ni, Zr & \cite{mendelev2012development} & \verb+eamfs-NiZr-Mendelev-2012+\\\hline
        Pb & \cite{zhou2004misfit} & \verb+eamalloy-Pb-Zhou-2004+\\\hline
        Pd & \cite{zhou2004misfit} & \verb+eamalloy-Pd-Zhou-2004+\\\hline
        Pt & \cite{zhou2004misfit} & \verb+eamalloy-Pt-Zhou-2004+\\\hline
        Ru & \cite{fortini2008asperity} & \verb+eamfs-Ru-Fortinin-2008+\\\hline
        Ta & \cite{li2003embedded} & \verb+eamalloy-Ta-Li-2003+\\
        & \cite{zhou2004misfit} & \verb+eamalloy-Ta-Zhou-2004+\\\hline
        Ti & \cite{ackland1992theoretical} & \verb+eamfs-Ti-Ackland-1992+\\
        & \cite{zhou2004misfit} & \verb+eamalloy-Ti-Zhou-2004+\\\hline
        U & \cite{smirnova2012interatomic} & \verb+eamalloy-U-Smirnova-2012+\\\hline
        W & \cite{zhou2004misfit} & \verb+eamalloy-W-Zhou-2004+\\\hline
        Zr & \cite{zhou2004misfit} & \verb+eamalloy-Zr-Zhou-2004+\\\hline
      \end{tabular}
    \end{center}
  }
\end{table}
\todo{Note that potential files by Zhou(2004)\cite{zhou2004misfit} are all single species files, which can be used as they are, but can also be combined. For further information on viability of these combinations and how to produce them, see \url{http://www.ctcms.nist.gov/potentials/Zhou04.html}. This is created and maintained by NIST Interatomic Potentials Repository\cite{arnold2011models} and in particlur is not developed by
Fraunhofer SCAI, nor is it sold by Fraunhofer SCAI or their
distribution in other ways restricted. Fraunhofer SCAI is not
responsible for functionality or maintenance of these potentials.}

\subsection{Tersoff}
\subsubsection*{Introduction}
The so called Tersoff type II potential for silicon was introduced in \cite{tersoff88a}. A slightly modified parameter set produces the so called Tersoff type III potential for
silicon  \cite{tersoff88b}. Specifically for C, Si and Ge a general Tersoff potential has been constructed in \cite{tersoff89}.

\subsubsection{Multicomponent Potential}
The energy is composed of
\begin{equation*}
  E=\sum_iE_i=\frac{1}{2}\sum_{i\neq j}\underbrace{f_C(r_{ij})\left(\chi_{Rij}f_R(r_{ij})+b_{ij}f_A(r_{ij})\right)}_{V_{ij}}
  ,
\end{equation*}
where the \textit{repulsive} and the \textit{attractive} terms
\begin{eqnarray*}
  f_R(r_{ij}) & = &A_{ij}\exp(-\lambda_{ij}r_{ij}) \\
  f_A(r_{ij}) & = &-B_{ij}\exp(-\mu_{ij}r_{ij}) \\
\end{eqnarray*}
and the smoothed cut off function
\begin{equation*}
  f_C(r_{ij}) = \left\{
    \begin{array}{ll}
      1  & r_{ij} < R_{ij}\\
      \frac{1}{2}+\frac{1}{2}\cos\left(\frac{\pi(r_{ij}-R_{ij})}{S_{ij}-R_{ij}}\right)  & R_{ij} < r_{ij} < S_{ij}\\
      0  & r_{ij} > S_{ij}
    \end{array}
  \right.
\end{equation*}
are symmetric in $i$ and $j$. An equivalent formulation is
\begin{equation*}
  E=\sum_{i<j}\underbrace{f_C(r_{ij})\left(\chi_{Rij}f_R(r_{ij})+\underbrace{\frac{(b_{ij}+b_{ji})}{2}}_{\tilde{b}_{ij}}f_A(r_{ij})\right)}_{\tilde{V}_{ij}}
\end{equation*}
In addition, the following holds:
\begin{eqnarray*}
  b_{ij} & = & \chi_{ij}(1+\beta_{ij}^{n_{ij}}\zeta_{ij}^{n_{ij}})^{-\frac{1}{2n_{ij}}}\\
  \zeta_{ij} & = & \sum_{k\neq i, j}f_C(r_{ik})\omega_{ijk}e^{\alpha_{ijk}^{m_{ijk}}(r_{ij}-r_{ik})^{m_{ijk}}}g(\theta_{ijk})\\
  g(\theta_{ijk}) & = & 1+\frac{c_{ik}^2}{d_{ik}^2}-\frac{c_{ik}^2}{d_{ik}^2+(h_{ik}-\cos\theta_{ijk})^2}
  .
\end{eqnarray*}
The mixture rules for the parameters are as follows;
\begin{equation}\label{equ:TersoffMixingRules}
  \begin{split}
    \lambda_{ij} & = \frac{\lambda_i+\lambda_j}{2} , \, \mu_{ij} = \frac{\mu_i+\mu_j}{2}\\
    A_{ij} & = \sqrt{A_iA_j} , \,
    B_{ij} = \sqrt{B_iB_j} , \,
    R_{ij} = \sqrt{R_iR_j} , \,
    S_{ij} = \sqrt{S_iS_j}  , \,
  \end{split}
\end{equation}
\begin{equation}\label{equ:TersoffMixingRules3}
  \omega_{ijk} = \omega_{ik}, \, \alpha_{ijk} = \alpha_{ik}, \, m_{ijk} = m_{ik} 
\end{equation}
and in \cite{tersoff89} it is set also $\alpha_{ijk}=\alpha_{ik}=0$, $m_{ijk}=m_{ik}=0$ and
\begin{equation}\label{equ:TersoffMixingRules2}
  \beta_{ij} = \beta_i , \,
  n_{ij} = n_{i} , \, 
  c_{ij} = c_i , \,
  d_{ij} = d_i , \,
  h_{ij} = h_i , \,
\end{equation}
Typically the special pair parameters $\chi_{ij}$ and $\omega_{ij}$ are symmetric in $i$ and $j$ as well (generally they are identical to $1$, unless specified differently). Furthermore, the common notation of $r_{ij}$ as the distance of $i$-$j$ and $\theta_{ijk}$ as the angle enclosed by  $i$-$j$ and $i$-$k$ is used.

For high energergetic simulations, one can modify the repulsive behaviour of the potential by
using instead of $V_{ij}$ the modified potential 
\begin{equation}\label{equ:TersoffZBLType2}
  \tilde{V}_{ij}(r) =  (1-F(r))V^{ZBL}(r)+F(r)V_{ij}(r),
\end{equation}
where $V^{ZBL}$ is the well-known Ziegler–Biersack–Littmark universal repulsive
potential \cite{ziegler1985stopping} and $F$ is the Fermi function
\begin{equation*}
  F(r)=\frac{1}{1+\exp(-b_f(r-r_f))}.
\end{equation*}
Alternatively one may just modufy the repulsive potential $f_R$ by 
\begin{equation}\label{equ:TersoffZBLType1}
  \tilde{f}_{R}(r) =  (1-F(r))V^{ZBL}(r)+F(r)f_{R}(r).
\end{equation}
Note that 
\begin{equation*}
  V^{ZBL}(r) = \frac{e^2}{4\pi\epsilon_0}\frac{Z_1 Z_2}{r} \phi\left(\frac{r}{a}\right)
\end{equation*}
with
\begin{equation*}
  a = \frac{0.8854 a_0}{Z_1^{0.23}+Z_2^{0.23}}
\end{equation*}
and
\begin{equation*}
  \phi(x) = 0.1818e^{-3.2 x}+0.5099e^{-0.9423 x}+0.2802e^{-0.4029 x}+0.02817e^{-0.2016 x}.
\end{equation*}
Here, $a_0$ is usually the bohradius, i.e. $0.529 \angstrom$. Note that the coulomb-force-constant \verb+epsilon0inv+ has to be specified, see also section \ref{sub:coulomb}.

% \subsubsection{Forces}
% As elsewhere, let \(\mathbf{r}_{ij}=\mathbf{x}_j-\mathbf{x}_i\) and thus \( -\nabla_{\mathbf{x}_i} \, r_{ij} = \frac{\mathbf{r}_{ij}}{r_{ij}}\). Furthermore:
% \begin{equation*}
%   \begin{split}
%     -\nabla \tilde{V}_{ij} = & \left[f'_C(r_{ij})(\chi_{Rij}f_R(r_{ij})+\tilde{b}_{ij}f_A(r_{ij}))+f_C(r_{ij})\left(f'_R(r_{ij})+\tilde{b}_{ij}f'_A(r_{ij})\right)\right](-\nabla r_{ij})  \\
%     &  + f_C(r_{ij})f_A(r_{ij})(-\nabla \tilde{b}_{ij})\\
%    -\nabla \tilde{b}_{ij} = & \frac{1}{2}( -\nabla b_{ij}-\nabla b_{ji}) \\
%    -\nabla b_{ij}  = & \chi_{ij}\left(-\frac{1}{2n_i}\right)(1+\beta_i^{n_i}\zeta_{ij}^{n_i})^{-\frac{1}{2n_i}-1}\beta_i^{n_i}n_i\zeta_{ij}^{n_i-1}(-\nabla \zeta_{ij})\\
%    = & -\frac{1}{2}\chi_{ij}(1+\beta_i^{n_i}\zeta_{ij}^{n_i})^{-\frac{1}{2n_i}-1}\beta_i^{n_i}\zeta_{ij}^{n_i-1}(-\nabla \zeta_{ij})\\
%     -\nabla \zeta_{ij}  = & \sum_{k\neq i,j}f'_C(r_{ik})\omega_{ik}g(\theta_{ijk})(-\nabla r_{ik})  \\
%     & + \sum_{k\neq i,j}f_C(r_{ik})\omega_{ik}\left(-\nabla g(\theta_{ijk})\right) \\
%     \left(-\nabla g(\theta_{ijk})\right) = & -\left(\frac{c_i}{d_i^2+(h_i-\cos\theta_{ijk})^2}\right)^22\left(h_i-\cos\theta_{ijk}\right)(-\nabla\cos\theta_{ijk})
%   \end{split}
% \end{equation*}
% Additionally we have
% \begin{eqnarray*}
%   \nabla_i\cos\theta_{ijk} & = & -\nabla_j\cos\theta_{ijk}-\nabla_k\cos\theta_{ijk}\\
%   \nabla_j\cos\theta_{ijk} & = & \frac{1}{D}(\mathbf{r}_{ik}-\frac{\langle \mathbf{r}_{ij}, \mathbf{r}_{ik}\rangle}{D}\frac{r_{ik}}{r_{ij}}\mathbf{r}_{ij})\\
%   \nabla_k\cos\theta_{ijk} & = & \frac{1}{D}(\mathbf{r}_{ij}-\frac{\langle \mathbf{r}_{ij}, \mathbf{r}_{ik}\rangle}{D}\frac{r_{ij}}{r_{ik}}\mathbf{r}_{ik})
% \end{eqnarray*}
% with \( D=r_{ij}r_{ik} \).



\subsubsection{User entries}
To set a Tersoff particle one should use:
\begin{lstlisting}
tersoff {
        tersoffparticle:        particle_type=C,        
                                A=1393.6,
                                B=346.74,
                                lambda=3.4879,
                                mu=2.2119,
                                beta=1.5724e-07,
                                n=0.72751,
                                c=38049,
                                d=4.3484,
                                h=-0.57058,
                                chiR=1.0,      
                                chi=1.0,       
                                omega=1.0,
                                alpha=0.0, 
                                m=0, 
                                R=1.8, 
                                S=2.1;
        };
\end{lstlisting}

To mix particle types, one may use:
\begin{lstlisting}
        tersoffmixit:           particle_type1=C,       
                                particle_type2=Si,      
                                chiR=1.0,      
                                chi=0.9776,       
                                omega=1.0,
                                alpha=0.0, 
                                m=0;
\end{lstlisting}
which applies the mixing rules (\ref{equ:TersoffMixingRules}), (\ref{equ:TersoffMixingRules2}) and (\ref{equ:TersoffMixingRules3}).

Alternative one may use:
\begin{lstlisting}
        tersoffoffdiag:         particle_type1=C,       
                                particle_type2=Si,
                                A=1597.311,             
                                B=395.145,      
                                lambda=2.9839,  
                                mu=1.97205, 
                                chiR=1.0,      
                                chi=1.0,       
                                omega=1.0,
                                alpha=0.0, 
                                m=0, 
                                R=2.21,                 
                                S=2.51;
\end{lstlisting}
which applies only the mixing rules (\ref{equ:TersoffMixingRules2}) and (\ref{equ:TersoffMixingRules3}).

Alternative one can use:
\begin{lstlisting}
        tersoffoffdiag2:        particle_type1=C,       
                                particle_type2=Si,      
                                A=1779.36144,   
                                B=225.189481,   
                                lambda=3.26563307,      
                                mu=1.76807421,  
                                beta=1.0,       
                                n=1.0,  
                                c=273987,       
                                d=180.314,      
                                h=-0.68,        
                                chiR=1.0,      
                                chi=1.0,       
                                omega=0.011877,
                                alpha=0.0, 
                                m=0,
                                R=2.2,  
                                S=2.6;
\end{lstlisting}
which just applies the mixing rules  (\ref{equ:TersoffMixingRules3}).

The parameters $\omega_{ijk}$, $\alpha_{ijk}$ and $m_{ijk}$ can also be specified for each triple by
\begin{lstlisting}
        tersofftriple:          particle_type1=C,       
                                particle_type2=Si,      
                                particle_type3=Si,  
                                omega=0.011877,
                                alpha=0.0, 
                                m=0;
\end{lstlisting}

The modified Tersoff/ZBL potential can be used by:
\begin{lstlisting}
        tersoffzbl:		particle_type1=Si,
	                        particle_type2=Si,	
                                type=2,
				Z1=14,			
                                Z2=14,			
                                a0=0.529,
				rf=0.95, 		
                                bf=14;
\end{lstlisting}
Here, to use variant (\ref{equ:TersoffZBLType2}) set \verb+type=2+ and
to use variant (\ref{equ:TersoffZBLType1}) set \verb+type=1+.
Here, $a_0$ is usually the bohradius, i.e. , i.e. $0.529 \angstrom$. Note that the coulomb-force-constant \verb+epsilon0inv+ has to bespecified, see also section \ref{sub:coulomb}.

Particle types may be set to any particle you specified in the particle section of this file.

\subsubsection{Parameters}
Examples of parameter sets for the Terosoff potential model are given in Table \ref{tab:Tersoff}.

\begin{table}
  \caption{Different parameter sets for the Tersoff potenial model}
  \label{tab:Tersoff}
  {\footnotesize
  \begin{center}
    \begin{tabular}{ccc}
      Elements & References & \texttt{.potentials} file \\\hline
      Al, As, Ga & \cite{nordlund2000strain} & \verb+tersoff-AlGaAs_2000+\\
      \hline
      Al, N & \cite{kioseoglou2008interatomic} & \verb+tersoff-AlN_2008+\\
      \hline
      Al, N, O & \cite{okeke2009molecular} & \verb+tersoff-AlNO_2009+\\
      & \cite{okeke2009molecular} & \verb+tersoff-AlNO_2009b+\\
      \hline
      As, In & \cite{hammerschmidt2008analytic} & \verb+tersoff-InAs_2008+\\
      \hline
      As, Ga & \cite{albe2002modeling} & \verb+tersoff-GaAs_2002+\\
             & \cite{hammerschmidt2008analytic} & \verb+tersoff-GaAs_2008+\\
             & \cite{fichthorn2011analytic} & \verb+tersoff-GaAs_2011+\\
      \hline
      As, Ga, In & \cite{nordlund2000strain} & \verb+tersoff-InGaAs_2000+\\
      \hline
      Au & \cite{backman2012bond}& \verb+tersoff-Au_2012+\\
      \hline
      B, C, N & \cite{matsunaga2000tersoff} & \verb+tersoff-BNC_2000+\\
      \hline
      B, N & \cite{moon2003molecular} &  \verb+tersoff-BN_2003+\\
      \hline
      B, N, O & \cite{okeke2009molecular} & \verb+tersoff-BNO_2009+\\
      & \cite{okeke2009molecular} & \verb+tersoff-BNO_2009b+\\
      \hline
      B, N, Si & \cite{matsunaga2001molecular} & \verb+tersoff-SiBN_2001+\\
      \hline
      Be, C, H & \cite{0953-8984-21-44-445002} & \verb+tersoff-BeCH_2009+\\
      \hline
      Be, H & \cite{0953-8984-21-44-445002} & \verb+tersoff-BeH_2009+\\
      \hline
      Be, W & \cite{bjorkas2010w} & \verb+tersoff-BeW_2010+\\
      \hline
      C & \cite{lindsay2010optimized} & \verb+tersoff-C_2010+\\
      \hline
      C, Fe & \cite{henriksson2009simulations} & \verb+tersoff-FeC_2009+\\
      \hline
      C, H & \cite{juslin2005analytical} & \verb+tersoff-CH_2005+\\
      C, H & \cite{juslin2005analytical,lindsay2010optimized} & \verb+tersoff-CH_2010+\\
      \hline
      C, H, W & \cite{juslin2005analytical} & \verb+tersoff-WCH_2005+\\
              & \cite{juslin2005analytical,erhart2005analytical} & \verb+tersoff-WCH_2005b+\\
      \hline
      C, Pt & \cite{albe2002modelingB}& \verb+tersoff-PtC_2002+\\
      \hline
      C, Si & \cite{tersoff89,tersoff1990erratum} & \verb+tersoff-SiC_1989+\\
      & \cite{tersoff1994chemical} &  \verb+tersoff-SiC_1994+\\
      & \cite{devanathan1998displacement} & \verb+tersoff-SiC_1998+\\
      & \cite{erhart2005analytical} & \verb+tersoff-SiC_2005+\\
      \hline
      Cu, Fe & \cite{hou2012analytic} & \verb+tersoff-FeCu_2012+\\
      \hline
      Er, H & \cite{peng2011bond} & \verb+tersoff-ErH_2011+\\
      \hline
      Fe & \cite{muller2007analytic} & \verb+tersoff-Fe_2007+\\
      \hline
      Fe, Pt & \cite{muller2007thermodynamics} & \verb+tersoff-FePt_2007+\\
      \hline
      Ga, N & \cite{nord2003modelling} & \verb+tersoff-GaN_2003+\\
      \hline
      Ga, N, O & \cite{okeke2009molecular} & \verb+tersoff-GaNO_2009+\\
      & \cite{okeke2009molecular} & \verb+tersoff-GaNO_2009b+\\
      \hline
      Ge, Si & \cite{tersoff89,tersoff1990erratum} & \verb+tersoff-SiGe_1989+\\
      \hline
      H, N, Si & \cite{de1999hydrogen} & \verb+tersoff-SiNH_2009+\\
      \hline
      H, W  & \cite{li2011modified} & \verb+tersoff-WH_2011+\\
      \hline
      In, N & \cite{kioseoglou2008interatomic} & \verb+tersoff-InN_2008+\\
      \hline
      In, N, O & \cite{okeke2009molecular} & \verb+tersoff-InNO_2009+\\
      & \cite{okeke2009molecular} & \verb+tersoff-InNO_2009b+\\
      \hline
      O & \cite{erhart2006analytic} & \verb+tersoff-O_2006+\\
      \hline
      O, Si & \cite{munetoh2007interatomic} & \verb+tersoff-SiO_2007+\\
      \hline
      O, Zn & \cite{erhart2006analytic} & \verb+tersoff-ZnO_2006+\\
      \hline
      Pt & \cite{albe2002modelingB}& \verb+tersoff-Pt_2002+\\
      \hline
      Si & \cite{tersoff1988new} & \verb+tersoff-SiC_1988+\\
         & \cite{erhart2005analytical} & \verb+tersoff-Si_2005+\\
         \hline
      Zn & \cite{erhart2006analytic} & \verb+tersoff-Zn_2006+\\
    \end{tabular}
  \end{center}
}
\end{table}
%Parameters for $B$ and $N$ may be found in  \cite{matsunaga01}. 
% Parameters for $Ga$ and $As$ have been published elsewhere.



%Modifications for the parameters of $Si$, $Ga$ and $As$ have been used in \cite{conrad98}.

%The potential has been extended to the Tersoff-Brenner-Potential for $C$-$H$ \cite{brenner90,brenner02} and $Si$-$H$ \cite{murty95,izumi}.

\subsection{Miscellaneous models}
There are miscellaneous combinations of potential terms for several
systems given in the potentials directory. For example a combination
of potential terms given in section \ref{sec:Tapered} is by Marian and
Gastreich in \cite{marian2000systematic} to model Si/B/N(H) compounds.

An overview is of miscellaneous potential models is given in Table
\ref{tab:MiscPot}.

\begin{table}
  \caption{Miscellaneous potential models given by combinations of various potential terms.}
  \label{tab:MiscPot}
  {\small
  \begin{center}
    \begin{tabular}{ccc}
      Elements & References & \texttt{.potentials} file \\\hline
      Si, B, N, H & \cite{marian2000systematic} & \verb+mg-SiBNH_2000+\\
      \hline
      Si, B, N & \cite{gastreich2003charged} & \verb+mg-SiBN_2003spline+\\
      &  & \verb+mg-SiBN_2003spme+\\
      \hline
    \end{tabular}
  \end{center}
}
\end{table}
 

\section{Bonded Potentials}
\todo{explicit range of sum}
\todo{up to date?}
\begin{eqnarray*}
E &=& E_{LJ} + E_C + E_{bonded} \\
E_{bonded} &=& E_b + E_{\theta} + E_{\phi} + E_{\psi} \\
E_b &=& \sum k_b (r -r_0)^2 \\
E_{\theta} &=& \sum k_{\theta} (\theta - \theta_0)^2 \\
E_{\phi} &=& \sum k_{\phi} ( 1 + \cos (n \phi - \delta )) \\
E_{\psi} &=& \sum k_{\psi} ( \psi - \psi_0)^2
\end{eqnarray*}

\todo{In comment}
% CHARMM: zusaetzlich h-bonds, constraints (SHAKE) von Laengen
% und dieder-Winkeln
\subsection{Bonds}\label{sse:bondedpotentials}
\subsubsection{Harmonic Bonds}
\begin{equation*}
U = k_B(r-r_0)^2
\end{equation*}
\emph{Note:} In this particular instance $k_B$ does NOT denote the Boltzmann constant, but the bond force constant.

\begin{lstlisting}
 bonds   {
        bond:   particle_type1=H,
                particle_type2=H,
                bond_type=harmonic,
                k_b=22208.534,
                r_0=0.28147059,
                k=0,
                e_0=0;
        };
\end{lstlisting}

\subsubsection{Morse Bonds}
\begin{equation*}
U = E_0 \cdot \left( \left[ 1- \exp^{-k(r-r_0)} \right]^2-1 \right)
\end{equation*}
\begin{lstlisting}
 bonds   {
        bond:   particle_type1=H,
                particle_type2=H,
                bond_type=morse,
                k_b=22208.534,
                r_0=0.28147059,
                k=0,
                e_0=0;
        };
\end{lstlisting}



\subsection{Angles}
\paragraph{Harmonic Angles}
\begin{equation*}
U = k_\theta\cdot(\theta-\theta_0)^2
\end{equation*}

\begin{lstlisting}
angles  {
        angle:  particle_type1=H,
                particle_type2=H,
                particle_type3=H,
                angle_type=harmonic,
                k_th=234.80792,
                theta_0=104.52,
                k_1=168.07514,
                k_2=92.444488,
                k=0,
                cos_theta_0=0,
                k_ub=792.20868,
                S_0=0.75323529;
        };
\end{lstlisting}

\paragraph{Harmonic cosine Angles}
\begin{equation*}
U = \frac{k}{2} \cdot\left( \cos(\theta) -\cos(\theta_0)\right)^2
\end{equation*}
\paragraph{Song Hi Lee Angles}
\begin{equation*}
U = k_1\cdot(\theta-\theta_0)^2 - k_2\cdot(\theta-\theta_0)^3
\end{equation*}
\paragraph{Harmonic Urey-Bradley Angles}
\begin{equation*}
U = k_\theta\cdot(\theta-\theta_0)^2 + k_{ub}\cdot(S-S_0)^2
\end{equation*}

\subsection{Diederpotential (Torsion)} \label{potentials:torsion}

These potentials apply to four-body structures of atoms. Proper torsions are those
which are used on ``linear`` configurations.
For non-linear four-body structures (e.g. three atoms bound to one atom in the center)
so called improper torsions are used. (see section \ref{potentials:improper_torsion})
\todo{torsion image?}

\subsubsection{Torsion cosine formula}
\begin{eqnarray*}
E_{\phi} &=&\sum_{i=1}^{mult} k_i ( 1 + \cos (n_i \phi - \delta_i )) \\
\end{eqnarray*}

with $1 \le n_i \le 6$, mult $< 6$, $\phi$ the angle between the plains spanned by
$(x_i, x_j, x_k)$ and $(x_j, x_k, x_l)$, for the case $(x_i, x_j, x_k, x_l)$ are
connected in a chain.

With $r_i = x_j - x_i, \quad r_j = x_k - x_j, \quad r_k = x_l - x_k$ we can write
\[
\cos (\phi) =
  \frac{ ( r_i \times r_j ) \cdot ( r_j \times r_k) }
     { | r_i \times r_j | | r_j  \times r_k | }
\]
\todo{Formula with scalar products, so user can decide it is the same?}\\

\bigbreak

\begin{lstlisting}
 torsions        {
        torsion:        particle_type1=H,
                        particle_type2=H,
                        particle_type3=H,
                        particle_type4=H,
                        torsion_type=cosine,
                        mult=5,
                        k_1=0,  n_1=0,  delta_1=0,
                        k_2=0,  n_2=0,  delta_2=0;
                        k_3=0,  n_3=0,  delta_3=0;
                        k_4=0,  n_4=0,  delta_4=0;
                        k_5=0,  n_5=0,  delta_5=0;
                 };
\end{lstlisting}

For the computation of the diederpotential and the differentiation the
angle phi does not always need to be computed. Singularities in the
derivative are caught by the program.


\subsubsection{Alternative Torsion (polynomial)}


\begin{equation}
U = k \sum_{i=0}^{5} a_i \cos{(\phi)}^i
\end{equation}


\begin{lstlisting}
 torsions        {
        torsion:        particle_type1=H,
                        particle_type2=H,
                        particle_type3=H,
                        particle_type4=H,
                        torsion_type=polynomial,
                        k=0,
                        a0=0,
                        a1=0,
                        a2=0,
                        a3=0,
                        a4=0,
                        a5=0,
                 };
\end{lstlisting}


\subsection{Improper Torsion} \label{potentials:improper_torsion}

Improper torsions apply to quadruples of atoms with a non-linear configuration
(e.g. three atoms connected to one in the center). Torsions for linear configurations
are handled in section \ref{potentials:torsion}

\todo{computation does not interest user!?}

\todo{but design of improper torsion does.}

\begin{eqnarray*}
E_{\psi} &=& k_{\psi} ( \psi - \psi_0)^2
\end{eqnarray*}

\begin{lstlisting}
 impropers       {
        improper:       particle_type1=H,
                        particle_type2=H,
                        particle_type3=H,
                        particle_type4=H,
                        improper_type=harmonic,
                        k_psi=85.384684,
                        psi_0=1;
        };
\end{lstlisting}

Equivalently we use the first terms of the Taylor series for small angles
for improper torsion terms to get a numerically stable computation:
\begin{eqnarray*}
E &\simeq& k \psi^2 \\
\frac{ \partial E}{\partial \cos\psi} &\simeq& 2 k
  \left( 1 + \frac{\psi^2}{6} \left(1 + \frac{7 \psi^2}{60} \right) \right) \\
\end{eqnarray*}
For large angle (starting about $6 \pi/180$) we simply compute the force by
\[ \frac{ \partial E}{\partial \cos\psi} =
    -2 k \frac{ \psi - \psi_0 }{ \sin\psi} \]
The derivative is done as for the diederpotential. Improper Torsion does not
apply to atoms in a chain, but considers a ``star'' of four atoms $(x_i, x_j,
x_k, x_l)$ with $x_i$ in center, $(x_j,x_k,x_l)$ each sharing a bond with $x_i$.
The angle $\psi$ is between the plains spanned by
$(x_i,x_j,x_k)$ and $(x_j,x_k,x_l)$.
In addition, some force field model consider improper torsion contributions for atoms not directly bonded.

\nocite{charmm83}
\nocite{charmm27b-online}

\subsubsection{Alternative Improper Torsion}
\begin{eqnarray*}
E_{\psi} &=& k_{\psi} \left(\sin( \psi - \psi_0)\right)^2
\end{eqnarray*}

\begin{lstlisting}
 impropers       {
        improper:       particle_type1=H,
                        particle_type2=H,
                        particle_type3=H,
                        particle_type4=H,
                        improper_type=squaredsine,
                        k_psi=20.0001,
                        psi_0=1;
        };
\end{lstlisting}




\section{Tapered Potentials}
\label{sec:Tapered}
Tapered potentials, similary to Lennard-Jones with splines, don't use a hard
cutoff but instead are multiplied with a decaying fifth order spline for
particle distances $r$: $x_i < r \leq x_o \enskip \wedge \enskip U(k) = 0,\  k \geq x_o$. This
way the energy is preserved despite the potential cutoff.

\subsection{BMHFT}
The non-coulombic term of the Born-Mayer-Huggins-Fumi-Tosi-Potential as used in \cite{tosifuminacl}. \todo{Cite actual source instead of ``usage example''?}

\begin{align*}
    U = A \cdot e^{B\,\left (\sigma-r\right )} - \frac{C}{r^6} - \frac{D}{r^8}
\end{align*}

\begin{lstlisting}[caption={Example taken from  NaCl test case using \texttt{kcalpermole} units.}]
    tbtapered_potentials {
        tbtaper: x_i = 20.0, x_o=23.0;
        tbtosifumi: particle_type1=Na, particle_type2=Cl, A=4.86167, B=3.1546, C=161.097, D=199.933, sigma=2.755;
    };
\end{lstlisting}


\subsection{Morse}
\label{subsec:NonBondMorse}
\todo{Double check potential (taken from tooltip):}
\begin{align*}
U = E_0 \cdot \left( \left[ 1- e^{-k(r-r_0)} \right]^2-1 \right)
\end{align*}

\begin{lstlisting}
 tbtapered_potentials    {
           tbtaper:        x_i=1.4705882,  x_o=1.7647059;
           tbmorse:        particle_type1=H,
                           particle_type2=H,
                           r_0=0.37428235,
                           k=10.06196,
                           e_0=364.60871;
           };
\end{lstlisting}


\subsection{Damped Dispersion}
\label{subsec:NonBondDampedDsipersion}
\todo{Double check potential (taken from tooltip):}
\begin{align*}
U = \frac{-C_b}{r^6\left( 1- e^{(-b_b\cdot r)} \sum_{k=0}^6\frac{(b_b\cdot r)^k}{k!} \right)}
\end{align*}
\begin{lstlisting}
 tbtapered_potentials    {
           tbtaper:               x_i=1.4705882,  x_o=1.7647059;
          dampeddispersion:       particle_type1=H,
                                  particle_type2=H,
                                  C_b=1.2712611,
                                  b_b=9.75375;
           };
\end{lstlisting}

\subsection{General}
\label{subsec:NonBondGeneral}
\paragraph{General Type I}
\todo{Double check potential (taken from tooltip):}
\begin{align*}
U = \frac{A}{r}\cdot \exp(\frac{-r}{\rho})
\end{align*}
\begin{lstlisting}
 tbtapered_potentials    {
                   tbtaper:                x_i=1.4705882,  x_o=1.7647059;
                   tbtaperedgeneral1:      particle_type1=H,
                                           particle_type2=H,
                                           A=1,
                                           rho=1;
           };
\end{lstlisting}

\paragraph{General Type II}
\todo{Double check potential (taken from tooltip):}
\begin{align*}
U = \frac{A}{r^2}\cdot \exp(\frac{-r}{\rho})-\frac{C}{r}
\end{align*}
\begin{lstlisting}
 tbtapered_potentials    {
                   tbtaper:                x_i=1.4705882,  x_o=1.7647059;
                   tbtaperedgeneral2:      particle_type1=H,
                                           particle_type2=H,
                                           A=0,
                                           rho=0,
                                           C=0;
           };
\end{lstlisting}

\paragraph{General Type III}
\begin{align*}
    U = A \cdot e^{B\,(\sigma - r)}
\end{align*}
\begin{lstlisting}
 tbtapered_potentials    {
                   tbtaper:                x_i=1.4705882,  x_o=1.7647059;
                   tbtaperedgeneral3:      particle_type1=H,
                                           particle_type2=H,
                                           A=0,
                                           B=0,
                                           sigma=0;
           };
\end{lstlisting}
\subsection{Buckingham}
\todo{Double check potential (taken from tooltip):}
\begin{align*}
U = A\cdot \exp(\frac{-r}{\rho})
\end{align*}
\begin{lstlisting}
 tbtapered_potentials    {
           tbtaper:        x_i=1.4705882,  x_o=1.7647059;
           tbbuckingham:   particle_type1=H,
                           particle_type2=H,
                           A=1,
                           rho=1;
           };
\end{lstlisting}

\subsection{BN 3 body}
\todo{Double check potential (taken from tooltip):}
\begin{align*}
U = k \cdot \exp\left(\frac{r_{ij}}{\rho_1}-\frac{r_{ik}}{\rho_2}\right)\cdot \frac{((\theta_0-\pi)^2 -(\theta-\pi)^2 )^2 }{8(\theta_0 -\pi)^2}
\end{align*}
\begin{lstlisting}
 tbtapered_potentials    {
           tbtaper:        x_i=1.4705882,  x_o=1.7647059;
           bn3body:        particle_type1=H,
                           particle_type2=H,
                           particle_type3=H,
                           bn3bodyentry=VESSAL,
                           k=2662095.9,
                           theta_0=118.864,
                           rho1=0.093416765,
                           rho2=0.093416765,
                           rmax1=0.79411765,
                           rmax2=0.79411765;
           };
\end{lstlisting}
