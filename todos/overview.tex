% Created 2015-05-04 Mon 15:29
\documentclass[11pt]{article}
\usepackage[utf8]{inputenc}
\usepackage[T1]{fontenc}
\usepackage{fixltx2e}
\usepackage{graphicx}
\usepackage{longtable}
\usepackage{float}
\usepackage{wrapfig}
\usepackage{rotating}
\usepackage[normalem]{ulem}
\usepackage{amsmath}
\usepackage{textcomp}
\usepackage{marvosym}
\usepackage{wasysym}
\usepackage{amssymb}
\usepackage{hyperref}
\tolerance=1000
\author{Pascal Huber}
\date{\today}
\title{System of units in tremologui}
\hypersetup{
  pdfkeywords={},
  pdfsubject={},
  pdfcreator={Emacs 24.4.1 (Org mode 8.2.10)}}
\begin{document}

\maketitle
\tableofcontents

\section{Description: Behaviour of tremologui with regard to the system of units.}
\label{sec-1}
In the following the behaviour of tremologui with respect to changes in the
project's system of units in described:
\subsection{Loading an existent project}
\label{sec-1-1}
This section describes the behaviour of tremologui when opening an existent
project using the "Open Project" button:
\subsubsection{No system of units declared in the project to be loaded:}
\label{sec-1-1-1}
If neither in the tremolo-file nor in the potentials-file any system of units is
present, tremologui aborts the load-process and displays an error message.
\subsubsection{System of units present in only one of the two files}
\label{sec-1-1-2}
In this case the user is asked if he/she would like to transfer the system of
units also to the file in which it is missing. If he accepts the system of units
is transfered immediately to the other file and the project is loaded. If he
declines, the project is loaded nevertheless.
\subsubsection{Different system of units}
\label{sec-1-1-3}
If the system of units in both files differ, then the load-process is aborted
and an error message is displayed.
\subsubsection{Invalid system of units}
\label{sec-1-1-4}
If in one of the files an invalid descriptor for the system of units is set
(e.g. \texttt{global: systemofunits=iamstupid}) then tremologui aborts the load-process
and issues an error message.
\subsubsection{Consistent system of units}
\label{sec-1-1-5}
If in both files the same system of units (and in the case of \texttt{custom} the same
parameters) are set, then the project is loaded without errors.
\subsection{Creating a new project}
\label{sec-1-2}
If a new project is created the system of units is set to \texttt{KCal per Mole} by
default and the system of units is written to both files (tremolo-file and
potentials-file) immediately.
\subsection{Changing the system of units and saving the current project}
\label{sec-1-3}
This section describes the behaviour of tremologui if the system of units of
the currently loaded project is changed within tremologui and these changes are
then saved to the project.
\subsubsection{Change system of units in "General parameters" tab}
\label{sec-1-3-1}
If the user changes the system of units in the "General parameters" tab of
tremologui and attempts to save the project afterwards, then tremologui issues
a warning messages. In this message the user is asked if he/she would like to
apply the system of units set in the \textbf{potentials}-file to the project.
If he/she choses "Apply" the system of units is reset to the value set in the
potentials-file. If he/she choses "Abort" the same message appears again (the
reason for this is how the save-methods in tremologui are written\ldots{}) and if
the user choses "Abort" once again an error message appears indicating that the
project was not saved.
This means: if the user wants to change the system of units he/she has to do
it in the "Generam parameters" tab \textbf{and} in the potentials file.
\subsubsection{Edit/change the potentials file}
\label{sec-1-3-2}
If the system of units in the potentials-file is changed (either by manually
editing the existent file or by loading a new file) and the user likes to
save the changes afterwards, then tremologui issues the same message as above,
i.e. the user is asked if he/she would like to "Apply" the system of units in
the potentials file to the whole project (in this case the system of units is
transferred immediately to the tremolo-file) or if he/she would like to "Abort"
the save-process (in this case the behaviour is the same as described in the
previous subsection).
\subsubsection{Saving if no system of units is set in the potentials-file}
\label{sec-1-3-3}
If no system of units is set in the potentials-file and the user attempts to
save the current project, he/she is asked if tremologui should transfer the
current system of units (as displayed in the "General parameters" tab) to the
potentials file. (If he/she declines this question the question is displayed
again - this is the same sort of bug as for the error message described in
two preceding sections).
\subsection{General remarks}
\label{sec-1-4}
If the one of the two files (tremolo-file or potentials-file) contains a keyword
like \texttt{systemofunits} at some strange position in the file (e.g. as part of the
file-path) the file will not be parsed.
% Emacs 24.4.1 (Org mode 8.2.10)
\end{document}