% Created 2014-10-29 Wed 11:04
\documentclass[a4paper]{article}
\usepackage[utf8]{inputenc}
\usepackage[T1]{fontenc}
\usepackage{fixltx2e}
\usepackage{graphicx}
\usepackage{longtable}
\usepackage{float}
\usepackage{wrapfig}
\usepackage{soul}
\usepackage{textcomp}
\usepackage{marvosym}
\usepackage{wasysym}
\usepackage{latexsym}
\usepackage{amssymb}
\usepackage{hyperref}
\tolerance=1000
\providecommand{\alert}[1]{\textbf{#1}}

\title{Füge Komfortfunktionen zu particle.c hinzu [0/7]}
\author{Pascal Huber}
\date{\today}
\hypersetup{
  pdfkeywords={},
  pdfsubject={},
  pdfcreator={Emacs Org-mode version 7.9.3f}}

\begin{document}

\maketitle

\setcounter{tocdepth}{3}
\tableofcontents
\vspace*{1cm}
\section{\textbf{TODO} \texttt{void *createLocalHessians(Particle *p, int numberOfHessians)}}
\label{sec-1}

\begin{itemize}
\item Allocate meomory for a double array of size \texttt{NDIMMAT x numberOfNeighbors}
  and assign it to the \texttt{localHessians} pointer of \texttt{p}.
\item Create a \texttt{trx\_htab} of size greater than \texttt{numberOfNeighbors} and assign it to the \texttt{hessianIndex} pointer of \texttt{p}.
\end{itemize}
\section{\textbf{TODO} void destroyParticleHessians(Particle *p)}
\label{sec-2}

\end{document}
