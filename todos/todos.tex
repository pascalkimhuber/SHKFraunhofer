% Created 2014-04-23 Wed 15:21
\documentclass[11pt]{article}
\usepackage[utf8]{inputenc}
\usepackage[T1]{fontenc}
\usepackage{fixltx2e}
\usepackage{graphicx}
\usepackage{longtable}
\usepackage{float}
\usepackage{wrapfig}
\usepackage{soul}
\usepackage{textcomp}
\usepackage{marvosym}
\usepackage{wasysym}
\usepackage{latexsym}
\usepackage{amssymb}
\usepackage{hyperref}
\tolerance=1000
\providecommand{\alert}[1]{\textbf{#1}}

\title{Aufgaben}
\author{Pascal Huber}
\date{\today}
\hypersetup{
  pdfkeywords={},
  pdfsubject={},
  pdfcreator={Emacs Org-mode version 7.9.3f}}

\begin{document}

\maketitle

\setcounter{tocdepth}{3}
\tableofcontents
\vspace*{1cm}
\begin{itemize}
\item Note taken on \textit{2014-04-16 Wed 15:57} \\
Besprechung mit Christian um \textit{2014-04-16 Wed 15:57}
    
    Zwei Aufgabenziele:
\begin{enumerate}
\item Kurzfristiges Ziel:
       Überarbeitung des ``potentials'' tabs
\begin{itemize}
\item Übernehme Elemente von ``Datafile'' tab
\item Obere 2/3 des Tabs: Editor-Fenster in das ein Potentials-file geladen werden soll
\begin{itemize}
\item der Editor soll die Funktionen ``open'' and ``save'' bereitstellen
\item open-button:
\begin{itemize}
\item soll erweiterbar sein (d.h. es sollen noch mögliche ``Suchfunktionen'' eingebunden werden können -> handler)
\item vorerst: einen beliebigen Ordner öffnen
\item nach .potential files filtern
\item nachdem der Nutzer ein file ausgewählt hat, soll eine Kopie des files in den Projekt-Ordner gelegt werden mit dem Namen ``PROJECTNAME.potentials''
\item falls nach Sicherung eines .potential-files ein weiteres file geöffnet wird, soll eine Warnung zum ``Datenverlust'' abgegeben werden.
\end{itemize}
\item save-button:
\begin{itemize}
\item speichert die Änderungen im .potentials file.
\end{itemize}
\end{itemize}
\item Stelle sicher, dass immer ein .validates-file erzeugt wird (notfalls auch leer (default)).
\end{itemize}
\item Langfristiges Ziel:
\begin{itemize}
\item Verifizieren von Ensembles
\item etwas mit Monte-Carlo Methoden (siehe folgende Links)
\begin{itemize}
\item \href{file:///home/huber/Work/literature/Possible Master topics/Combining molecular dynamics with Monte Carlo simulations.pdf}{Combining molecular dynamics and Monte Carlo methods}
\item \href{file:///home/huber/Work/literature/Possible Master topics/Uniform-acceptance force-bias Monte Carlo method with time scale to study solid-state diffusion.pdf}{Uniform-acceptance force-bias Monte Carlo method}
\end{itemize}
\end{itemize}
\end{enumerate}
\end{itemize}

\section{\textbf{TODO} Create new branch}
\label{sec-1}

  
\section{\textbf{TODO} Determine which files have to be modified}
\label{sec-2}
\subsection{Files for Potential-tab}
\label{sec-2-1}

potential.cpp                
potential.h                  
potentialparameter$_{\mathrm{data}}$.cpp  
potentialparameter$_{\mathrm{data}}$.h    
potentialparametertable.cpp  
potentialparametertable.h
potentialparameterui.cpp
potentialparameterui.h
potentialparameteruisuper.cpp
potentialparameteruisuper.h
potentialparameterwidget.cpp
potentialparameterwidget.h
\subsection{Files for Datafile-tab}
\label{sec-2-2}

parametertextedit.h
parametertextedit.cpp
\subsection{Description of parametertextedit.h and parametertextedit.cpp}
\label{sec-2-3}

\begin{itemize}
\item Two classes
\begin{enumerate}
\item class textEditStatusBar : public QWidget
\item class parameterTextEdit : public QMainWindow
\end{enumerate}
\end{itemize}
\subsubsection{Description of textEditStatusBar class}
\label{sec-2-3-1}


Implements the status bar in the ``Datafile''-tab (just at the lower end of the window). 

The status bar consists of 4 different frames (QFrame) containing each a QLabel with
\begin{itemize}
\item the current position of the cursor
\item a indicator that shows if the file has been changed
\item the type of the file
\item the name of the file
\end{itemize}

The layout is managed by a QHBoxLayout for the statusbar itself and for different GridLayouts for each of the frames. 

The following slots are implemented:
\begin{itemize}
\item changeCurrentPos
\item setFilename
\item setEditMode
\item languageChange
\end{itemize}
\subsubsection{Description of parameterTextEdit class}
\label{sec-2-3-2}


Implements a texteditor for datafiles in the ``Datafile''-tab. It is essentially a QMainWindow with a QTextEdit. Moreover the class has a textEditStatusBar as child object.

The class contains the following children:
\begin{itemize}
\item QTextEdit for the texteditor window
\item QToolBar a toolbar for the texteditor
\item QActions for all buttons on the toolbar
\item QStrings to save the parameter file and a suffix
\item textEditStatusBar
\item bools for different indications
\end{itemize}

The layout is managed by only one QVBoxLayout. 

There are only a few method implemented in the class: 
\begin{itemize}
\item bool isSaved() const // returns true if the file is saved
\item void setSavedStatus(bool status) // sets the status of the file
\item void fileLoad(QString File) // loads the file that is supposed to be edited
\item void renewTextEdit2() // ?
\end{itemize}

Slots:
\begin{itemize}
\item void fileSave() // saves the file
\item void reloadfile(QString parameterFileIn) // reload the file
\item void clearBuffer(bool ask=true) // ?
\item void setLoadingfileStatus(bool status) // ?
\item void enableFileSaveAction(void) // ?
\item void changeCurrentPos (void) // ?
\item void fileOpen() // open new file
\item fileSaveAs() // save file as
\item textEditChanged // action if textEdit has been changed
\item virtual void languageChange() // set language
\end{itemize}

Signals:
\begin{itemize}
\item void newSaveState() // is send if the save-state of the file changes
\end{itemize}

\end{document}
