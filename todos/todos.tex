% Created 2014-07-22 Tue 17:22
\documentclass[11pt]{article}
\usepackage[utf8]{inputenc}
\usepackage[T1]{fontenc}
\usepackage{fixltx2e}
\usepackage{graphicx}
\usepackage{longtable}
\usepackage{float}
\usepackage{wrapfig}
\usepackage{soul}
\usepackage{textcomp}
\usepackage{marvosym}
\usepackage{wasysym}
\usepackage{latexsym}
\usepackage{amssymb}
\usepackage{hyperref}
\tolerance=1000
\providecommand{\alert}[1]{\textbf{#1}}

\title{todos}
\author{Pascal Huber}
\date{\today}
\hypersetup{
  pdfkeywords={},
  pdfsubject={},
  pdfcreator={Emacs Org-mode version 7.9.3f}}

\begin{document}

\maketitle

\setcounter{tocdepth}{3}
\tableofcontents
\vspace*{1cm}
\begin{itemize}
\item Note taken on \textit{2014-07-22 Tue 17:11} \\
Aufgabenbeschreibung:

     Im Zuge einer Software zur Ionen-Migration (Berechnung über die Poisson-Nernst-Planck Gleichung) soll ein script geschrieben werden, dass als Input Diffusionswerte, chemisches Potential und erwartete Fehler entgegennimmt und anschließend den elektrischen Fluß bestimmt.
     Dazu soll das Optimierungsprogramm DAKOTA (Design Analysis Kit for Optimization and Terascale Applications) von den SNL verwendet werden.

     Meine Aufgabe ist es nun
\begin{enumerate}
\item Mache mich Dakota vertraut
\begin{itemize}
\item Lese das gesamte Manual (Version 5.4).
\item Schaue, was für das obige Problem wichtig sein kann.
\end{itemize}
\end{enumerate}
Poisson-Nernst-Planck Gleichung

     $\partial$$_t$ c = $\nabla$ [D ($\nabla$ c + $\alpha$ $\nabla$ $\phi$ + $\beta$ c $\nabla$ $\mu$ )]

     $\Delta$ $\phi$ = $\sum$$_i$ z$_i$ c$_i$
\end{itemize}

\end{document}
