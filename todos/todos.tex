% Created 2014-01-20 Mon 18:14
\documentclass[11pt]{article}
\usepackage[utf8]{inputenc}
\usepackage[T1]{fontenc}
\usepackage{graphicx}
\usepackage{longtable}
\usepackage{float}
\usepackage{wrapfig}
\usepackage{soul}
\usepackage{amssymb}
\usepackage{hyperref}


\title{Manual fixes}
\author{Pascal Huber}
\date{20 January 2014}

\begin{document}

\maketitle

\setcounter{tocdepth}{3}
\tableofcontents
\vspace*{1cm}
   \texttt{DEADLINE:} \textit{2014-01-23 Thu}\newline

\begin{itemize}
\item Created branch \textbf{manual-fix} in tremolo directory.
\end{itemize}
\section{\textbf{TODO} Fix all things in the Tremolo-Tutorial that are obvious}
\label{sec-1}

     \texttt{SCHEDULED:} \textit{2014-01-20 Mon}\newline

\subsection{Gemachte Aenderungen:}
\label{sec-1.1}

\begin{description}
\item [Page 108] \emph{Visuals shall be created every 5 time units or 10 iteration steps, whereas the particle data shall be written every 500 time units or 10 iteration steps.} \\
                 ``Visuals \textbf{(argon.vis.\##\##.[xyz,pdb,data])} shall be created \ldots{}''
\item [Page 109] \emph{The optimized particle positions are written to argon.data.999.} \\
                 ``The optimized particle positions are written to argon.data.999 \textbf{9}.''
\item [Page 110] \emph{For this example we can still ignore the extra lines.} \\
                  ``For this example we can still ignore the extra lines \textbf{at the top of the file}.''
\item [Page 111] \emph{Take a look at the velocity distribution in the argon.histogram file} \\
                  ``*Plot a histogram displaying the velocity distribution contained in the argon.histogram file*.''
\item [Page 112] \emph{1.4 An alternative: The Nose-Hoover-thermostat} \\
                  ``1.4 An alternative: The Nosé-Hoover-thermostat''
\item [Page 112] \emph{Thus, when using the Nosé-Hoover-thermostat considerations with respect to equilibration are imperative.} \\
                  ``Thus, when using the Nosé-Hoover-thermostat, considerations with respect to equilibration are imperative.'' (Komma hinzugefuegt)
\item [Page 113] \emph{We begin by modifying the the optimization of the sample \ldots{}} \\
                  ``We begin by modifying \textbf{the} optimization of the sample\ldots{}''
\item [Page 114] \emph{Again we work in the =argon.parameter= file and enter the thermostat after (or in place of) the thermostat:} \\
                  Again we work in the \texttt{argon.parameter} file and enter the \textbf{barostat} after (or in place of) the thermostat: \\
\item [Page 115] \emph{So go ahead and change the domain in the appropriate line} \\
                  ``So go ahead and change the domain in the appropriate line \textbf{of the argon.parameters} file.''
\item [Page 115] \emph{Change the extpressure value.} \\
                  ``Change the \textbf{Pressure} value.''
\item [Page 118] \emph{\ldots{} but this time additionally to the external pressure we also support a custom stress tensor, \ldots{}} \\
                   ``\ldots{} but this time additionally to the external pressure we also support a custom stress tensor $\sigma$, \ldots{}''
                   Die Spannung mit $\sigma$ zu bezeichnen, scheint Standard zu sein: \href{http://en.wikipedia.org/wiki/Stress_%28mechanics%29}{Wikipedia(Stress)}
\item [Page 119] \emph{The strain is defined as the length change \ldots{}} \\
                   ``The strain $\epsilon$ is defined as the length change \ldots{}''
\item [Page 120] \emph{Add a 100 [t] relaxation time at the beginning ($\sigma$ = 0).} \\
                   ``Add a 100 [t] relaxation time ($\sigma$ = 0) at the beginning \textbf{of the simulation by inserting an additional line in the =stresstensor= section of the =graphene.parameters= file.}''
\item [Page 121] \emph{As stated earlier the potential should not be cut off but has to to fit the linked cell structure of the domain.}
                   ``As stated earlier the potential should not be cut off but has \textbf{to} fit the linked cell structure of the domain.''
\item [Page 125] \emph{TODO} entfernt (kann keinen Buchstabendreher entdecken).
\end{description}
\subsection{\textbf{TODO} Noch zu machende Aenderungen:}
\label{sec-1.2}

\begin{description}
\item [Page 110] \emph{For the integration of the particle trajectories we choose a standard verlet algorithm with a time step of 0.005 custom time units.} \\
                  
                  Kann nicht beurteilen, ob deltaT=0.5e-3 oder deltaT=5e-3 richtig ist. 
                  Output fuer die Geschwindigkeitsverteilung in Kapitel 1.2:\\
                  \href{file:///home/huber/Work/todos/pics/argon.histogram.png}{Geschwindigkeitsverteilung bei deltaT=0.5e-3}\\
                  \href{file:///home/huber/Work/todos/pics/argon.short.histogram.png}{Geschwindigkeitsverteilung bei deltaT=5e-3}
\item [Page 114] Letzter Punkt in Exercises 14.5.1: Optimierung ist nicht moeglich fuer \texttt{constraint=isotropic}, wenn nur eine /secondary axis= geaendert wird.
\item [Page 114] Das Listing unterscheidet sich vom File im \texttt{/tutorials/}-Ordner. Beispielsweise: \texttt{ensemble=NPE} im Listing und \texttt{ensemble=NPT} im File im Ornder \texttt{6sim\_npt}
\item [Page 115] \emph{Should you attempt to start it right away, you will receive an error message.} \\
                   Ich erhalte keine Fehlermeldung, allerdings die Warnung: \texttt{Attention: BOX statement in .data file overwrites domain size data in .parameters file suggest correcting moment and angular moment with Max.-Boltz. distribution}
\item [Page 115] Dritter Punkt in Exercises 14.6.1: \emph{In the} \texttt{constraintmap} \emph{change one of the secondary axis entries. Check and compare the new box values.} \\
                  Hier muss auch der \texttt{type} der \texttt{constraintmap} geaendert werden. Darueber hinaus muessen bei den sekundaeren Achsen immer Paare geaendert werden.
\item [Page 122] Die Simulation ist fehlerhaft: Die meisten Partikel verlassen das Gebiet. Das \texttt{*.generalmeas} file enthaelt nur Nullen.
\item [Page 129] Erhalte die folgenden Fehlermeldungen: 
                   \texttt{Error: Cannot open file: /home/neuen/tremolo/tutorial/12eam.external} \\
                   \texttt{Error: Cannot open file: /home/neuen/tremolo/tutorial/12eam.exttypes} \\
                   In den Kapiteln 14.1 bis 14.11 habe ich diese Meldungen nicht finden koennen.
\end{description}
\section{\textbf{DONE} Check 1.6: Box parameters: What happens for different values in *.parameters and *.data?}
\label{sec-2}

     \texttt{SCHEDULED:} \textit{2014-01-20 Mon} \texttt{CLOSED:} \textit{2014-01-20 Mon 18:11}\newline
     Erhalte die folgende Fehlermeldung, wenn die Werte der Box in \texttt{*.parameters} und in \texttt{*.data} nicht uebereinstimmen: \\
     \texttt{Attention: BOX statement in .data file overwrites domain size data in .parameters file suggest correcting moment and angular moment with Max.-Boltz. distribution}.

\section{\textbf{TODO} [ ] Check 1.2: Check 0.0005 and 0.005.}
\label{sec-3}

     \texttt{SCHEDULED:} \textit{2014-01-20 Mon}\newline

\section{\textbf{DONE} Create new branch for that.}
\label{sec-4}

     \texttt{SCHEDULED:} \textit{2014-01-20 Mon} \texttt{CLOSED:} \textit{2014-01-19 Sun 13:12}\newline

\end{document}