\documentclass{scrartcl}

\usepackage[utf8]{inputenc}
\usepackage{amsmath}

\begin{document}

\section*{Übersicht zur Hesse-Matrix für Paar-Potentiale}

Gegeben sei ein System von \(N\) Partikeln mit Ortskoordinaten \(\vec{r}_1, \dots, \vec{r}_N\). Das dazugehörige Potential \(V(\vec{r}_1, \dots, \vec{r}_N)\) kann als Summe von Paar-Potentialen \(V_{r_i, r_j}\) geschrieben werden:
\[V(\vec{r}_1, \dots, \vec{r}_N) = \sum_{i=1}^N \sum_{j=i+1}^N V_{r_i, r_j}(\vec{r}_i, \vec{r}_j). \]
Hierbei wird von symmetrischen Paar-Potentialen ausgegangen.

\subsection*{Kraftberechnung}
\label{sec:kraftberechnung}

Die aus dem Potential resultierende Kraft auf ein Paritkel \(\vec{r}_i\) ergibt sich durch den Gradienten von \(V\) bzgl. \(\vec{r}_i\), d.h.
\[\vec{F}_{r_i} (\vec{r}_1, \dots, \vec{r}_N) = \nabla_{\vec{r}_i} V(\vec{r}_1, \dots, \vec{r}_N) = \nabla_{\vec{r}_i} \sum_{j=1, j\neq i}^N V_{r_i, r_j}(\vec{r}_i, \vec{r}_j), \]
wobei in der letzten Gleichung die Symmetrie der Paar-Potentiale ausgenutzt wurde.
Also
\[ \vec{F}_{r_i} (\vec{r}_1, \dots, \vec{r}_N) = \left( \partial_{r_{i_1}} \sum_{j=1, j\neq i}^N V_{r_i, r_j}(\vec{r}_i, \vec{r}_j), \dots, \partial_{r_{i_3}} \sum_{j=1, j\neq i}^N V_{r_i, r_j}(\vec{r}_i, \vec{r}_j) \right).\]

\subsection*{Hesse-Matrix}
\label{sec:hesse-matrix}

Die Hesse-Matrix des Gesamtsystems bestehend aus den \(N\) Partikeln \(\vec{r}_1, \dots, \vec{r}_N\) ist eine \(3N \times 3N\)-Matrix der Form
\[H_V(\vec{r}_1, \dots, \vec{r}_N) = \left( \partial_{\vec{r}_i} \partial_{\vec{r}_j} V(\vec{r}_1, \dots, \vec{r}_N) \right)_{i,j = 1}^N. \]

Dabei ist \(\partial_{\vec{r}_i} \partial_{\vec{r}_j} V(\vec{r}_1, \dots, \vec{r}_N)\) eine \(3\times 3\)-Matrix der Form
\[
\begin{pmatrix}
\partial_{r_{i_1}}\partial_{r_{j_1}} V(\dots) & \dots & \partial_{r_{i_1}}\partial_{r_{j_1}} V(\dots) \\
\vdots & \ddots & \vdots \\
\partial_{r_{i_1}}\partial_{r_{j_1}} V(\dots) & \dots & \partial_{r_{i_1}}\partial_{r_{j_1}} V(\dots)
\end{pmatrix}
\]

Hierbei kann ausgenutzt werden, dass \(V\) als Summe von
Paar-Potentialen dargestellt werden kann.

\subsubsection*{1. Fall \(i \neq j\)}
\label{sec:1.-fall}
Hier gilt
\begin{align*}
\partial_{\vec{r}_i}\partial_{\vec{r}_j} V(\vec{r}_1, \dots, \vec{r}_N) &= \partial_{\vec{r}_i} \left( \sum_{k=1, k\neq j}^N \partial_{\vec{r}_j} V_{r_j r_k}(\vec{r}_j, \vec{r}_k) \right) \\
&= \partial_{\vec{r}_i}\partial_{\vec{r}_j} V_{r_j r_i}(\vec{r}_i, \vec{r}_j).
\end{align*}

\subsubsection*{2. Fall \(i = j\)}
\label{sec:2.-fall}
In diesem Fall gilt:
\begin{align*}
\partial_{\vec{r}_i}\partial_{\vec{r}_i} V(\vec{r}_1, \dots, \vec{r}_N) &= \partial_{\vec{r}_i} \left( \sum_{j=1, j\neq i}^N \partial_{\vec{r}_i} V_{r_i r_j}(\vec{r}_i, \vec{r}_j) \right) \\
&= \sum_{j=1, j\neq i}^N \partial_{\vec{r}_i} \partial_{\vec{r}_i} V_{r_i r_j}(\vec{r}_i, \vec{r}_j).
\end{align*}

\subsubsection*{Zusammenfassung}
\label{sec:zusammenfassung}

Insgesamt lässt sich die Hesse-Matrix des Gesamtsystems also wie folgt
darstellen:
\[H_V(\vec{r}_1, \dots, \vec{r}_N) =
\left( \partial_{\vec{r}_i} \partial_{\vec{r}_j} V(\vec{r}_1, \dots,
  \vec{r}_N) \right)_{i,j = 1}^N \]
mit den \(3\times 3\)-Matrizen
\[
\partial_{\vec{r}_i}\partial_{\vec{r}_j} V(\vec{r}_1, \dots, \vec{r}_N) =
\begin{cases}
  \partial_{\vec{r}_i}\partial_{\vec{r}_j} V_{r_j r_i}(\vec{r}_i,
  \vec{r}_j) & \text{, falls } i \neq j \\
  \partial_{\vec{r}_i} \partial_{\vec{r}_i} \sum_{j=1, j\neq i}^N
  V_{r_i r_j}(\vec{r}_i, \vec{r}_j) & \text{, sonst.}
\end{cases}
\]

\subsection*{Implementierung}
\label{sec:implementierung}

Die Hesse-Matrix des Systems wird für die Implementierung in die
lokalen \(3\times 3\)-Matrizen
\(\partial_{\vec{r}_i}\partial_{\vec{r}_j} V(\vec{r_1}, \dots,
\vec{r_N})\) aufgeteilt.
Jedes \texttt{Particle struct} \(r_i\) speichert dabei die Matrizen
\[
\partial_{\vec{r}_i} \partial_{\vec{r}_j} V_{r_i r_j} \text{für } j=1,
\dots, N, j\neq i \quad \text{ und }
\quad \partial_{\vec{r}_i} \partial_{\vec{r}_i} V_{r_i r_j} \text{ für
} j = 1, \dots, N, j\neq i.
\]

Die dafür benötigten Datenstrukturen sind:
\begin{itemize}
\item Ein Array \texttt{localHessians} der Größe \(N\times (3\times
  3)\)\footnote{Aufgrund der Lokalität der verwendeten Potentiale
    dürfte auch ein deutlich kleineres Array in diesem Fall genügen.}. Dies soll die lokalen Hesse-Matrizen speichern.
\item Eine Map \texttt{hessianIndex<particle, localIndex>}, welche die
  Indizes der Partikel auf den richtigen Eintrag in
  \texttt{localHessians} mapped.
\end{itemize}



\end{document}

%%% Local Variables:
%%% mode: latex
%%% TeX-master: "."
%%% End:
