% Created 2015-02-26 Thu 16:49
\documentclass[11pt]{article}
\usepackage[utf8]{inputenc}
\usepackage[T1]{fontenc}
\usepackage{fixltx2e}
\usepackage{graphicx}
\usepackage{longtable}
\usepackage{float}
\usepackage{wrapfig}
\usepackage{rotating}
\usepackage[normalem]{ulem}
\usepackage{amsmath}
\usepackage{textcomp}
\usepackage{marvosym}
\usepackage{wasysym}
\usepackage{amssymb}
\usepackage{hyperref}
\tolerance=1000
\author{Pascal Huber}
\date{\today}
\title{test}
\hypersetup{
  pdfkeywords={},
  pdfsubject={},
  pdfcreator={Emacs 24.4.1 (Org mode 8.2.10)}}
\begin{document}

\maketitle
\tableofcontents

\section{Tests of the Lennard-Jones Hessian implementation in tremolo \textit{<2015-01-21 Wed>}}
\label{sec-1}

\subsection{Overview: Lennard-Jones potential}
\label{sec-1-1}
Set
\begin{equation}
  \label{eq:1}
  V(r) =  4 \varepsilon \left( (\frac{\sigma}{r})^{12} -
    (\frac{\sigma}{r})^{6} \right) = 4 \varepsilon \left( R^{12} -
    R^6 \right),  \qquad \text{ with } R = \frac{\sigma}{r},
\end{equation}
and
\begin{equation}
  \label{eq:2}
  r(\boldp, \boldq) = \lVert \boldq - \boldp \rVert =
  \sqrt{\sum_{i=1}^d r_i^2} \qquad \text{ with } r_i \coloneqq (q_i - p_i).
\end{equation}

\subsubsection{Partial derivatives of r}
\label{sec-1-1-1}
We have (\(r = r(\boldp, \boldq)\)):
\begin{align}
\partial_{p_i} r(\boldp, \boldq) &= - \frac{r_i}{r},  \\
\partial_{q_j} r(\boldp, \boldq) &=  \frac{r_j}{r},  \\
\partial_{q_i}\partial_{p_i} r(\boldp, \boldq) &= - \partial_{p_i}\partial_{p_i} r(\boldp, \boldq) =  \frac{r_i^2}{r^3} - \frac{1}{r},  \\
\partial_{q_j}\partial_{p_i} r(\boldp, \boldq) &= - \partial_{p_j}\partial_{p_i} r(\boldp, \boldq) =  \frac{r_ir_j}{r^3}, \\
\end{align}

\subsubsection{Derivatives of the Lennard-Jones potential}
\label{sec-1-1-2}
We have
\begin{align}
  V'(r) &= \frac{24 \varepsilon}{r} R^6 \left(1 - 2 R^6\right) \\
  V''(r) &= \frac{24 \varepsilon}{r^2} R^6 \left( 26 R^6 - 7 \right).
\end{align}

\subsubsection{Lennard-Jones forces and Hessians}
\label{sec-1-1-3}
We have
\begin{align}
  \partial_{p_i} V(r(\boldp, \boldq)) &= - \frac{24 \varepsilon}{r^2} R^6 \left(1 - 2 R^6\right)r_i \\
  \partial_{q_i}\partial_{p_i} V(r(\boldp, \boldq)) &= - \partial_{p_i}\partial_{p_i} V(r(\boldp, \boldq)) = \frac{24 \varepsilon}{r^4} R^6 \left(8 - 28 R^6\right)r_i^2 - \frac{24 \varepsilon}{r^2} R^6 \left(1 - 2 R^6\right) \\
  \partial_{q_j}\partial_{p_i} V(r(\boldp, \boldq)) &= - \partial_{p_j}\partial_{p_i} V(r(\boldp, \boldq)) = \frac{24 \varepsilon}{r^4} R^6 \left(8 - 28 R^6\right)r_ir_j
\end{align}

\subsection{Test parameters}
\label{sec-1-2}
Summary of the test parameters. The directory to the parameter files for tremolo can be found \href{file:///home/huber/Sandbox/testHessians/}{here}.
\begin{itemize}
\item sigma = epsilon = 1
\item cellrcut = 12
\item edge length of the cube: 80
\item delta$_{\text{T}}$ = 5.0e-3
\item endtime = 5.0e-1
\item outvis T$_{\text{Delta}}$ = 1.0e-2
\end{itemize}
The test distances at which the particles are initially situated are given by:
\begin{itemize}
\item r$_{\text{1}}$ = 1 sigma
\item r$_{\text{2}}$ = r$_{\text{m}}$ = 2$^{\text{(1/6)}}$ sigma
\item r$_{\text{3}}$ = 3/2 sigma
\item r$_{\text{4}}$ = 4 sigma
\item r$_{\text{5}}$ = 20 (zero interaction)
\end{itemize}

\subsection{Without hessian tag}
\label{sec-1-3}
If the tag <hessians> does not exist, no files \texttt{<projectname>.xxxx.hessians} are created.
If the tag <hessians> exists but the option \texttt{measure=off} is set, no files \texttt{<projectname>.xxxx.hessians} are created.
If the tag <hessians> exists and an invalid option is set, then tremolo displays an error message and aborts the simulation.

\subsection{Test for two particles}
\label{sec-1-4}
Note that the local Hessians for a single particle $\partial$$_{\text{pp}}$ and for two different particles $\partial$$_{\text{pq}}$ only differ by a sign.
\subsubsection{Particles aligned in x-direction, no start velocities}
\label{sec-1-4-1}
For two particles aligned in x-direction one expects:
\begin{itemize}
\item diagonal matrices for all local Hessians (since all entries of the form r$_{\text{i}}$ r$_{\text{j}}$ should vanish if i $\neq$ j)
\item the second and third diagonal entry should be equal
\end{itemize}
\begin{enumerate}
\item Summary
\label{sec-1-4-1-1}
As expected one obtains for all radii diagonal matrices for which the second and third diagonal entries are equal.
For r$_{\text{2}}$ only the first entry is no equal to zero which is expected since the other two diagonal entries are given by the force value which in this case is zero as the particles are already at the energy minimum.
For r$_{\text{5}}$ no entries are calculated since the distance between the particles is larger than r$_{\text{cut}}$.
\item r$_{\text{1}}$
\label{sec-1-4-1-2}
\begin{verbatim}
# ATOMDATA Id x=3 u=3 type
1       40.0    40.0    40.0    0.0     0.0     0.0     Argon
2       41.0    40.0    40.0    0.0     0.0     0.0     Argon
\end{verbatim}
At all time steps one obtains diagonal matrices as expected.
In the following the absolute values of the local Hessian $\partial$$_{\text{p}_{\text{1}} \ \text{p}_{\text{1}}}$ are given:
\begin{description}
\item[{0000}] r = 1, diag(456, 24, 24)
\item[{0050}] r = 1.8156, diag(1.275256, 0.191907, 0.191907)
\item[{0100}] r = 2.16214, diag(0.338966, 0.049267, 0.049267)
\end{description}
\item r$_{\text{2}}$
\label{sec-1-4-1-3}
\begin{verbatim}
# ATOMDATA Id x=3 u=3 type
1       40.0          40.0    40.0      0.0     0.0     0.0     Argon
2       41.122462048  40.0    40.0      0.0     0.0     0.0     Argon
\end{verbatim}
Since r$_{\text{2}}$ is the distance of the minimal potential energy the distance between the two particles remains the same during the whole simulation.
As expected one obtains at all times the same matrix which consists of only one single non-zero entry at the (1,1)-position.
The other two diagonal entries dissapear since they consists basically of the force between the two particles which vanishes in this case.
In the following the absolute values of the local Hessian $\partial$$_{\text{p}_{\text{1}} \ \text{p}_{\text{1}}}$ are given:
\begin{description}
\item[{0000}] r = 1.122462048, diag(57.146438, 0, 0)
\item[{0050}] r = 1.122462048, diag(57.146438, 0, 0)
\item[{0100}] r = 1.122462048, diag(57.146438, 0, 0)
\end{description}
\item r$_{\text{3}}$
\label{sec-1-4-1-4}
\begin{verbatim}
# ATOMDATA Id x=3 u=3 type
1       40.0  40.0    40.0      0.0     0.0     0.0     Argon
2       41.5  40.0    40.0      0.0     0.0     0.0     Argon
\end{verbatim}
As for r$_{\text{1}}$ one obtains at all time steps diagonal matrices.
In the following the absolute values of the local Hessian $\partial$$_{\text{p}_{\text{1}} \ \text{p}_{\text{1}}}$ are given:
\begin{description}
\item[{0000}] r = 1.5, diag(4.41759, 0.772019, 0.772019)
\item[{0050}] r = 1,13286, diag(46.886467, 0.476441, 0.476441)
\item[{0100}] r = 1.45254, diag(5.125041, 0.953188, 0.953188)
\end{description}
\item r$_{\text{4}}$
\label{sec-1-4-1-5}
\begin{verbatim}
# ATOMDATA Id x=3 u=3 type
1       40.0  40.0    40.0      0.0     0.0     0.0     Argon
2       44.0  40.0    40.0      0.0     0.0     0.0     Argon
\end{verbatim}
Same situation as for r$_{\text{3}}$.
In the following the absolute values of the local Hessian $\partial$$_{\text{p}_{\text{1}} \ \text{p}_{\text{1}}}$ are given:
\begin{description}
\item[{0000}] r = 4, diag(0.002561, 0.000366, 0.000366)
\item[{0050}] r = 3.99926, diag(0.002565, 0.000367, 0.000367)
\item[{0100}] r = 3.99706, diag(0.002576, 0.000368, 0.000368)
\end{description}
\item r$_{\text{5}}$
\label{sec-1-4-1-6}
\begin{verbatim}
# ATOMDATA Id x=3 u=3 type
1       40.0  40.0    40.0      0.0     0.0     0.0     Argon
2       60.0  40.0    40.0      0.0     0.0     0.0     Argon
\end{verbatim}
In this case no interaction between the two particles can happen, since the distance larger than r$_{\text{cut}}$. Thus at all times one gets empty hessians file of the following form:
\begin{verbatim}
# time 0.000000e+00
# particle_id1   coord1  particle_id2    coord2  hessian_entry
\end{verbatim}
Note that in the case of r = r$_{\text{cut}}$ Hessians are calculated and one does not obtain empty hessians files but files which contain all only zero matrices!
\end{enumerate}
\subsubsection{Particles aligned in y-direction, no start velocities}
\label{sec-1-4-2}
For two particles aligned in y-direction one expects:
\begin{itemize}
\item diagonal matrices for all local Hessians (since all entries of the form r$_{\text{i}}$ r$_{\text{j}}$ should vanish if i $\neq$ j)
\item the first and third diagonal entry should be equal
\end{itemize}
\begin{enumerate}
\item Summary
\label{sec-1-4-2-1}
One obtains the analogous results as for the case of x-aligned particles.
\item r$_{\text{1}}$
\label{sec-1-4-2-2}
\begin{verbatim}
# ATOMDATA Id x=3 u=3 type
1       40.0    40.0    40.0    0.0     0.0     0.0     Argon
2       40.0    41.0    40.0    0.0     0.0     0.0     Argon
\end{verbatim}
At all time steps one obtains diagonal matrices as expected.
In the following the absolute values of the local Hessian $\partial$$_{\text{p}_{\text{1}} \ \text{p}_{\text{1}}}$ are given:
\begin{description}
\item[{0000}] r = 1, diag(24, 456, 24)
\item[{0050}] r = 1.8156, diag(0.191907, 1.275256, 0.191907)
\item[{0100}] r = 2.16214, diag(0.049267, 0.338966, 0.049267)
\end{description}
\item r$_{\text{2}}$
\label{sec-1-4-2-3}
\begin{verbatim}
# ATOMDATA Id x=3 u=3 type
1       40.0    40.0          40.0      0.0     0.0     0.0     Argon
2       40.0    41.122462048  40.0      0.0     0.0     0.0     Argon
\end{verbatim}
Since r$_{\text{2}}$ is the distance of the minimal potential energy the distance between the two particles remains the same during the whole simulation.
As expected one obtains at all times the same matrix which consists of only one single non-zero entry at the (2,2)-position.
The other two diagonal entries dissapear since they consists basically of the force between the two particles which vanishes in this case.
In the following the absolute values of the local Hessian $\partial$$_{\text{p}_{\text{1}} \ \text{p}_{\text{1}}}$ are given:
\begin{description}
\item[{0000}] r = 1.122462048, diag(0, 57.146438, 0)
\item[{0050}] r = 1.122462048, diag(0, 57.146438, 0)
\item[{0100}] r = 1.122462048, diag(0, 57.146438, 0)
\end{description}
\item r$_{\text{3}}$
\label{sec-1-4-2-4}
\begin{verbatim}
# ATOMDATA Id x=3 u=3 type
1       40.0  40.0    40.0      0.0     0.0     0.0     Argon
2       40.0  41.5    40.0      0.0     0.0     0.0     Argon
\end{verbatim}
As for r$_{\text{1}}$ one obtains at all time steps diagonal matrices.
In the following the absolute values of the local Hessian $\partial$$_{\text{p}_{\text{1}} \ \text{p}_{\text{1}}}$ are given:
\begin{description}
\item[{0000}] r = 1.5, diag(0.772019, 4.41759, 0.772019)
\item[{0050}] r = 1,13286, diag(0.476441, 46.886467, 0.476441)
\item[{0100}] r = 1.45254, diag(0.953188, 5.125041, 0.953188)
\end{description}
\item r$_{\text{4}}$
\label{sec-1-4-2-5}
\begin{verbatim}
# ATOMDATA Id x=3 u=3 type
1       40.0  40.0    40.0      0.0     0.0     0.0     Argon
2       40.0  44.0    40.0      0.0     0.0     0.0     Argon
\end{verbatim}
Same situation as for r$_{\text{3}}$.
In the following the absolute values of the local Hessian $\partial$$_{\text{p}_{\text{1}} \ \text{p}_{\text{1}}}$ are given:
\begin{description}
\item[{0000}] r = 4, diag(0.000366, 0.002561, 0.000366)
\item[{0050}] r = 3.99926, diag(0.000367, 0.002565, 0.000367)
\item[{0100}] r = 3.99706, diag(0.000368, 0.002576, 0.000368)
\end{description}
\item r$_{\text{5}}$
\label{sec-1-4-2-6}
\begin{verbatim}
# ATOMDATA Id x=3 u=3 type
1       40.0  40.0    40.0      0.0     0.0     0.0     Argon
2       40.0  60.0    40.0      0.0     0.0     0.0     Argon
\end{verbatim}
In this case no interaction between the two particles can happen, since the distance larger than r$_{\text{cut}}$. Thus at all times one gets empty hessians file of the following form:
\begin{verbatim}
# time 0.000000e+00
# particle_id1   coord1  particle_id2    coord2  hessian_entry
\end{verbatim}
Note that in the case of r = r$_{\text{cut}}$ Hessians are calculated and one does not obtain empty hessians files but files which contain all only zero matrices!
\end{enumerate}
\subsubsection{Particles aligned in z-direction, no start velocities}
\label{sec-1-4-3}
For two particles aligned in z-direction one expects:
\begin{itemize}
\item diagonal matrices for all local Hessians (since all entries of the form r$_{\text{i}}$ r$_{\text{j}}$ should vanish if i $\neq$ j)
\item the first and second diagonal entry should be equal
\end{itemize}
\begin{enumerate}
\item Summary
\label{sec-1-4-3-1}
One obtains the analogous results as for the case of x-aligned and y-aligned particles.
\item r$_{\text{1}}$
\label{sec-1-4-3-2}
\begin{verbatim}
# ATOMDATA Id x=3 u=3 type
1       40.0    40.0    40.0    0.0     0.0     0.0     Argon
2       40.0    40.0    41.0    0.0     0.0     0.0     Argon
\end{verbatim}
At all time steps one obtains diagonal matrices as expected.
In the following the absolute values of the local Hessian $\partial$$_{\text{p}_{\text{1}} \ \text{p}_{\text{1}}}$ are given:
\begin{description}
\item[{0000}] r = 1, diag(24, 24, 456)
\item[{0050}] r = 1.8156, diag(0.191907, 0.191907,  1.275256)
\item[{0100}] r = 2.16214, diag(0.049267, 0.049267, 0.338966)
\end{description}
\item r$_{\text{2}}$
\label{sec-1-4-3-3}
\begin{verbatim}
# ATOMDATA Id x=3 u=3 type
1       40.0    40.0    40.0           0.0      0.0     0.0     Argon
2       40.0    40.0    41.122462048   0.0      0.0     0.0     Argon
\end{verbatim}
Since r$_{\text{2}}$ is the distance of the minimal potential energy the distance between the two particles remains the same during the whole simulation.
As expected one obtains at all times the same matrix which consists of only one single non-zero entry at the (3,3)-position.
The other two diagonal entries dissapear since they consists basically of the force between the two particles which vanishes in this case.
In the following the absolute values of the local Hessian $\partial$$_{\text{p}_{\text{1}} \ \text{p}_{\text{1}}}$ are given:
\begin{description}
\item[{0000}] r = 1.122462048, diag(0, 0, 57.146438)
\item[{0050}] r = 1.122462048, diag(0, 0, 57.146438)
\item[{0100}] r = 1.122462048, diag(0, 0, 57.146438)
\end{description}
\item r$_{\text{3}}$
\label{sec-1-4-3-4}
\begin{verbatim}
# ATOMDATA Id x=3 u=3 type
1       40.0  40.0    40.0      0.0     0.0     0.0     Argon
2       40.0  40.0    41.5      0.0     0.0     0.0     Argon
\end{verbatim}
As for r$_{\text{1}}$ one obtains at all time steps diagonal matrices.
In the following the absolute values of the local Hessian $\partial$$_{\text{p}_{\text{1}} \ \text{p}_{\text{1}}}$ are given:
\begin{description}
\item[{0000}] r = 1.5, diag(0.772019, 0.772019, 4.41759)
\item[{0050}] r = 1,13286, diag(0.476441, 0.476441, 46.886467)
\item[{0100}] r = 1.45254, diag(0.953188, 0.953188, 5.125041)
\end{description}
\item r$_{\text{4}}$
\label{sec-1-4-3-5}
\begin{verbatim}
# ATOMDATA Id x=3 u=3 type
1       40.0  40.0    40.0      0.0     0.0     0.0     Argon
2       40.0  40.0    44.0      0.0     0.0     0.0     Argon
\end{verbatim}
Same situation as for r$_{\text{3}}$.
In the following the absolute values of the local Hessian $\partial$$_{\text{p}_{\text{1}} \ \text{p}_{\text{1}}}$ are given:
\begin{description}
\item[{0000}] r = 4, diag(0.000366, 0.000366, 0.002561)
\item[{0050}] r = 3.99926, diag(0.000367, 0.000367, 0.002565)
\item[{0100}] r = 3.99706, diag(0.000368, 0.000368, 0.002576)
\end{description}
\item r$_{\text{5}}$
\label{sec-1-4-3-6}
\begin{verbatim}
# ATOMDATA Id x=3 u=3 type
1       40.0  40.0    40.0      0.0     0.0     0.0     Argon
2       40.0  40.0    60.0      0.0     0.0     0.0     Argon
\end{verbatim}
In this case no interaction between the two particles can happen, since the distance larger than r$_{\text{cut}}$. Thus at all times one gets empty hessians file of the following form:
\begin{verbatim}
# time 0.000000e+00
# particle_id1   coord1  particle_id2    coord2  hessian_entry
\end{verbatim}
Note that in the case of r = r$_{\text{cut}}$ Hessians are calculated and one does not obtain empty hessians files but files which contain all only zero matrices!
\end{enumerate}
\subsubsection{Particles aligned along a diagonal, no start velocities}
\label{sec-1-4-4}
For two (initially stationary) particles which are aligned along a space diagonal one can expect:
\begin{itemize}
\item all diagonal entries of the local Hessians are equal
\item all non-diagonal entries of the local Hessians are equal
\end{itemize}
\begin{enumerate}
\item Summary
\label{sec-1-4-4-1}
As expected all computed local Hessians have the mentioned symmetric properties, such that all diagonal entries on the one hand and all non-diagonal entries on the other hand are equal.
In contrast to the previous cases no entries are equal to zero which is also quite reasonable, since the difference vector has no zero-entries.
The values of the entries seem to be correct as well.
\item r$_{\text{1}}$
\label{sec-1-4-4-2}
\begin{verbatim}
# ATOMDATA Id x=3 u=3 type
1       40.0            40.0            40.0            0.0     0.0     0.0     Argon
2       40.577350269    40.577350269    40.577350269    0.0     0.0     0.0     Argon
\end{verbatim}
The local Hessian $\partial$$_{\text{p}_{\text{1}} \ \text{p}_{\text{1}}}$ for the first particle at time 0 is given by\\
[136.000001 160.000001 160.000001]\\
[160.000001 136.000001 160.000001]\\
[136.000001 160.000001 136.000001]
The other three local Hessians differ only in the sign. As expected all diagonal entries are equal as are the non-diagonal ones.
The difference between the diagonal and non-diagonal entries is 24 which is exactly the norm of the force between the particles.
In the following the local Hessians $\partial$$_{\text{p}_{\text{1}} \ \text{p}_{\text{1}}}$ are given
\begin{description}
\item[{0000}] r = 1, (136.000001 160.000001 160.000001, 160.000001 136.000001 160.000001, 136.000001 160.000001 136.000001)
\item[{0050}] r = 1.815587618, (-0.297147 -0.489054 -0.489054, -0.489054 -0.297147 -0.489054, -0.489054 -0.489054 -0.297147)
\item[{0100}] r = 2.162136344, (-0.080144 -0.129411 -0.129411, -0.129411 -0.080144 -0.129411, -0.129411 -0.129411 -0.080144)
\end{description}
\item r$_{\text{2}}$
\label{sec-1-4-4-3}
\begin{verbatim}
# ATOMDATA Id x=3 u=3 type
1       40.0            40.0            40.0            0.0     0.0     0.0     Argon
2       40.648053766    40.648053766    40.648053766    0.0     0.0     0.0     Argon
\end{verbatim}
The local Hessian $\partial$$_{\text{p}_{\text{1}} \ \text{p}_{\text{1}}}$ for the first particle at time 0 is given by\\
[19.048812 19.048812 19.048812]\\
[19.048812 19.048812 19.048812]\\
[19.048812 19.048812 19.048812]\\
All entries are equal. Since the force at r$_{\text{2}}$ vanishes the second summand in the term for the diagonal entries of the
local Hessians dissappears and the remaining term is equal to the formula for the non-diagonal entries of the local Hessians.
In the following the local Hessians $\partial$$_{\text{p}_{\text{1}} \ \text{p}_{\text{1}}}$ are given
\begin{description}
\item[{0000}] r = 1.122462048, (19.048812 19.048812 19.048812, 19.048812 19.048812 19.048812, 19.048812 19.048812 19.048812)
\item[{0050}] r = 1.122462048, (19.048812 19.048812 19.048812, 19.048812 19.048812 19.048812, 19.048812 19.048812 19.048812)
\item[{0100}] r = 1.122462048, (19.048812 19.048812 19.048812, 19.048812 19.048812 19.048812, 19.048812 19.048812 19.048812)
\end{description}
\item r$_{\text{3}}$
\label{sec-1-4-4-4}
\begin{verbatim}
# ATOMDATA Id x=3 u=3 type
1       40.0            40.0            40.0            0.0     0.0     0.0     Argon
2       40.866025404    40.866025404    40.866025404    0.0     0.0     0.0     Argon
\end{verbatim}
The local hessian $\partial$$_{\text{p}_{\text{1}} \ \text{p}_{\text{1}}}$ for the first particle at time 0 is given by\\
[-0.957852 -1.729871 -1.729871]\\
[-1.729871 -0.957852 -1.729871]\\
[-1.729871 -1.729871 -0.957852]\\
The situation is similar as for r$_{\text{1}}$, i.e. the diagonal entries and the non-diagonal entries each are equal which is quite expected.
In the following the local Hessians $\partial$$_{\text{p}_{\text{1}} \ \text{p}_{\text{1}}}$ are given
\begin{description}
\item[{0000}] r = 1.5, (-0.957852 -1.729871 -1.729871, -1.729871 -0.957852 -1.729871, -1.729871 -1.729871 -0.957852)
\item[{0050}] r = 1.132865151, (15.946450 15.470009 15.470009, 15.470009 15.946450 15.470009, 15.470009 15.470009 15.946450)
\item[{0100}] r = 1.452549769, (-1.072889 -2.026076 -2.026076, -2.026076 -1.072889 -2.026076, -2.026076 -2.026076 -1.072889)
\end{description}
\item r$_{\text{4}}$
\label{sec-1-4-4-5}
\begin{verbatim}
# ATOMDATA Id x=3 u=3 type
1       40.0            40.0            40.0            0.0     0.0     0.0     Argon
2       42.309401077    42.309401077    42.309401077    0.0     0.0     0.0     Argon
\end{verbatim}
The local hessian $\partial$$_{\text{p}_{\text{1}} \ \text{p}_{\text{1}}}$ for the first particle at time 0 is given by\\
[-0.000610 -0.000976 -0.000976]\\
[-0.000976 -0.000610 -0.000976]\\
[-0.000976 -0.000976 -0.000610]\\
In this case too, the local Hessians maintain their symmetry properties as in the other cases.
In the following the local Hessians $\partial$$_{\text{p}_{\text{1}} \ \text{p}_{\text{1}}}$ are given
\begin{description}
\item[{0000}] r = 4, (-0.000610 -0.000976 -0.000976, -0.000976 -0.000610 -0.000976, -0.000976 -0.000976 -0.000610)
\item[{0050}] r = 3.999270674, (-0.000611 -0.000977 -0.000977, -0.000977 -0.000611 -0.000977, -0.000977 -0.000977 -0.000611)
\item[{0100}] r = 3.997070969, (-0.000613 -0.000981 -0.000981, -0.000981 -0.000613 -0.000981, -0.000981 -0.000981 -0.000613)
\end{description}
\item r$_{\text{5}}$
\label{sec-1-4-4-6}
\begin{verbatim}
# ATOMDATA Id x=3 u=3 type
1       40.0            40.0            40.0            0.0     0.0     0.0     Argon
2       51.547005384    51.547005384    51.547005384    0.0     0.0     0.0     Argon
\end{verbatim}
As expected no local Hessians are computed since there is no interaction between the two particles as r$_{\text{5}}$ > r$_{\text{cut}}$.
Thus all hessians files look like
\begin{verbatim}
# time 5.000000e-01
# particle_id1   coord1  particle_id2    coord2  hessian_entry
\end{verbatim}
\end{enumerate}
\subsubsection{Particle fly-by/swing-by}
\label{sec-1-4-5}
\begin{enumerate}
\item Description
\label{sec-1-4-5-1}
Two particles are positioned at zero-interaction distance from each other (p at (40.0, 40.0, 40.0) and q at (20.0, 45.0, 40.0)).
The first particle (p) is stationary while the second one has an initial velocity ((40.0, 0.0, 0.0)) such that it passes the
first particle at a certain distance.
The .data file is given by
\begin{verbatim}
# ATOMDATA Id x=3 u=3 type
1       40.0    40.0    40.0    0.0     0.0     0.0     Argon
2       20.0    45.0    40.0   40.0     0.0     0.0     Argon
\end{verbatim}
The simulation can be devided into three phases:
\begin{description}
\item[{zero-interaction phase}] (t = 0 until t = 0.22728\ldots{}) During this time the Particles have a distance which is larger than the r$_{\text{cut}}$ (12).
No interaction can be expected and therefore empty hessians-files.
\item[{interaction phase}] (For t = 0.22728 until t = 0.77272\ldots{}) During the second phase the Particles are near enought to each other
and one can expect non-zero Hessians. Since the z-coordinate of both Particles stays equal during the whole simulation
four of the nine entries of the local Hessians should be zero, namely the entries (1,3), (3,1), (2,3), (3,2).
Furthermore one can expect that the entries grow until time t = 0.5 and decrease afterwards.
\item[{zero-interaction phase}] (t = 0.77272 until t = 1.0) For the remaining time the particles have again zero-interaction distance.
Hence one should obtain empty hessians-files.
\end{description}
\item Test results
\label{sec-1-4-5-2}
As expected the hessians-files test.0000.hessians, \ldots{}, test.0023.hessians and test.0078.hessians, \ldots{}, test.0100.hessians are empty (zero-interaction).
The remaining files show the expected local Hessians. E.g. for time t = 0.40 (and up to signs also t = 0.60):
\begin{verbatim}
# time 4.000000e-01
# particle_id1   coord1  particle_id2    coord2  hessian_entry
1       0       1       0       -0.000018
1       0       1       1       0.000033
1       0       1       2       0.000000
1       1       1       0       0.000033
1       1       1       1       -0.000033
1       1       1       2       -0.000000
1       2       1       0       0.000000
1       2       1       1       -0.000000
1       2       1       2       0.000008
\end{verbatim}
One can see that the entries (1,3), (2,3), (3,1) and (3,2) are equal to zero. For t = 0.50 one obtains
\begin{verbatim}
# time 5.000000e-01
# particle_id1   coord1  particle_id2    coord2  hessian_entry
1       0       1       0       0.000061
1       0       1       1       -0.000000
1       0       1       2       -0.000000
1       1       1       0       -0.000000
1       1       1       1       -0.000430
1       1       1       2       -0.000000
1       2       1       0       -0.000000
1       2       1       1       -0.000000
1       2       1       2       0.000061
\end{verbatim}
In this case only the diagonal entries are non-zero since the Particles are aligned in y-direction.
\end{enumerate}
\subsubsection{Particle collision}
\label{sec-1-4-6}
\begin{enumerate}
\item Description
\label{sec-1-4-6-1}
Two particles are positioned at zero-interaction distance from each other (p at (40.0, 40.0, 40.0) and q at (20.0, 40.0, 40.0)).
The first particle (p) is stationary while the second one has an initial velocity ((40.0, 0.0, 0.0)) such that it will collide
with the first particle at time t=0.50.
The .data file is given by
\begin{verbatim}
# ATOMDATA Id x=3 u=3 type
1       40.0    40.0    40.0    0.0     0.0     0.0     Argon
2       20.0    40.0    40.0   40.0     0.0     0.0     Argon
\end{verbatim}
This simulation can be devided into three phases:
\begin{description}
\item[{zero-interaction phase}] (t = 0 until t = 0.20) During this time the particles have a distance which is larger than the r$_{\text{cut}}$ (12).
No interaction can be expected and therefore empty hessians-files.
\item[{interaction phase}] (t = 0.20 until t = 0.57) In this phase the abosolute values of the local Hessian entries increase first with time until the collision at approximately t = 0.5.
Afterwards they decrease since the second particle bounces back while the other one is accelerated in x-direction.
Since in this simulation the particles stay aligned in x-direction one can expect diagonal matrices for the local Hessians.
\item[{zero-interaction phase}] (t = 0.58 until t = 1.0) For the remaining time the particles have again zero-ineteraction distance.
\end{description}
\item Test results
\label{sec-1-4-6-2}
As expected the files test.0000.hessians, \ldots{}, test.0020.hessians and test.0057.hessians, \ldots{}, test.0100.hessians are empty.
All other files display diagonal matrices. For t = 0.48 one gets for the first local Hessian:
\begin{verbatim}
# time 4.800000e-01
# particle_id1   coord1  particle_id2    coord2  hessian_entry
1       0       1       0       13019.502641
1       0       1       1       0.000000
1       0       1       2       0.000000
1       1       1       0       0.000000
1       1       1       1       -935.941320
1       1       1       2       0.000000
1       2       1       0       0.000000
1       2       1       1       0.000000
1       2       1       2       -935.941320
\end{verbatim}
\end{enumerate}
\subsection{Test for three particles}
\label{sec-1-5}
\subsubsection{Three particles in an equilateral triangle}
\label{sec-1-5-1}
\begin{enumerate}
\item General description
\label{sec-1-5-1-1}
Three particles are positioned in such a way that they form an equilateral triangle with side length r = r$_{\text{1}}$, r$_{\text{2}}$, r$_{\text{3}}$, r$_{\text{4}}$ and r$_{\text{5}}$.
Only the Lennard-Jones potential is activated during the simulation, i.e. the entire potential is given by:
\[
V(p_1, p_2, p_3) = V_{12}(p_1, p_2) + V_{13}(p_1, p_3) + V_{23}(p_2, p_3),
\]
where V$_{\text{ij}}$ denotes the pair-potential between the particles p$_{\text{i}}$ and p$_{\text{j}}$.
Knowing this, one can deduce the following formulas for the 9 possible local Hessians:
\begin{itemize}
\item H$_{\text{11}}$ = $\partial$$_{\text{1}}$ $\partial$$_{\text{1}}$ V$_{\text{12}}$ + $\partial$$_{\text{1}}$ $\partial$$_{\text{1}}$ V$_{\text{13}}$
\item H$_{\text{22}}$ = $\partial$$_{\text{2}}$ $\partial$$_{\text{2}}$ V$_{\text{12}}$ + $\partial$$_{\text{2}}$ $\partial$$_{\text{2}}$ V$_{\text{23}}$
\item H$_{\text{33}}$ = $\partial$$_{\text{3}}$ $\partial$$_{\text{3}}$ V$_{\text{13}}$ + $\partial$$_{\text{3}}$ $\partial$$_{\text{3}}$ V$_{\text{23}}$
\item H$_{\text{12}}$ = $\partial$$_{\text{1}}$ $\partial$$_{\text{2}}$ V$_{\text{12}}$
\item H$_{\text{21}}$ = $\partial$$_{\text{2}}$ $\partial$$_{\text{1}}$ V$_{\text{12}}$
\item H$_{\text{13}}$ = $\partial$$_{\text{1}}$ $\partial$$_{\text{3}}$ V$_{\text{13}}$
\item H$_{\text{31}}$ = $\partial$$_{\text{3}}$ $\partial$$_{\text{1}}$ V$_{\text{13}}$
\item H$_{\text{23}}$ = $\partial$$_{\text{2}}$ $\partial$$_{\text{3}}$ V$_{\text{23}}$
\item H$_{\text{32}}$ = $\partial$$_{\text{3}}$ $\partial$$_{\text{2}}$ V$_{\text{23}}$
\end{itemize}
Hence, the mixed local Hessians can be expected to be similar to the 2 particles case whereas the "diagonal" local Hessians are sums of such expressions.
\end{enumerate}
% Emacs 24.4.1 (Org mode 8.2.10)
\end{document}