\chapter{Parallelization}
\label{parallelization}

One of the key strength of Tremolo-X is the parallelization.
Naturally there are some associated parameters, which you can set.

The most basic decision is the number of processes to be started.
They are organized by a division of the simulation domain along the
three major axes. The total number of processes is then the product
of the number of processes in x direction times the number of processes
in y direction times the number of processes in z direction.

There are some restrictions on the choice of processes:
Each process must at least contain one cell, thus the choice of the 
number of processes must at least satisfy the inequality:\\
$\frac{edge\_length}{number\_processes} > lc\_rcut,$\\
where $lc\_rcut$ is the cut radius (edge length) of the linked cells.

A prudent choice of this parameter deserves some thought, as both a good and
bad choice may reflect in the performance of the code. The range of the used potentials and the domain size and shape
should be considered.

\bigbreak
In the parameter file, the processor numbers are set as follows:
\begin{lstlisting}
 n1=<n1>       number of processes in x direction
 n2=<n2>       number of processes in y direction
 n3=<n3>       number of processes in z direction
 all=<all>     product of n1 n2 n3
\end{lstlisting}

It may be overridden with the command line switch -p <n1>,<n2>,<n3>, as documented in chapter \ref{running_tremolo}.

\bigbreak

After the division is established, cells are distributed via (dynamic) load-balancing. You can choose a number of parameters
to fit the use to your liking, apart from switching it on and off, you may specify intervals, after which the balancing is 
reexamined. In addition you may also choose, whether the load is measured linear or quadratic (with respect to the particle number assigned to a process).



% proc: n1=<n1>, n2=<n2>, n3=<n3>, all=<all>;
% n1=<n1>               [1] number of processes in x direction
% n2=<n2>               [1] number of processes in y direction
% n3=<n3>               [1] number of processes in z direction
% all=<all>             [1] product of n1 n2 n3

% loadbal: wf=<weight_func>, Lb_t=<Lb_t>, Lb_delta=<Lb_delta>, Lb_step=<Lb_step>;
% wf=<weight_func>      off | linear | quadratic
%                       weight function to use for determining the work count
% Lb_t=<Lb_t>           [t] start time of first load balancing step
% Lb_delta=<Lb_delta>   [t] time span to pass before the next load balancing step
% Lb_step=<Lb_step>     [1] number of optimization steps between two load balancing steps
%parallelization {
%                proc:   n1=1,   n2=1,   n3=1,   all=1;
%                loadbal:        wf=quadratic,   Lb_t=0.000000e+00,      Lb_delta=2.046167e+13,  Lb_step=1;
%                };
