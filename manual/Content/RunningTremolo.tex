\chapter{Running Tremolo}
\label{running_tremolo}

Tremolo-X allows two distinct modes of operation: Control from a command line, which allows the user to take advantage of scripts for task automation, and control from a graphical user interface, which provides an interactive environment and relieves the user of the burden to handle the parameter syntax.

\section{Command line}

Assuming, the Tremolo-X installation path belongs to the \$PATH environment of your system,
the most basic command to call (serial) tremolo is 
\begin{lstlisting}
 tremolo project.tremolo
\end{lstlisting}

In order to call the parallel version, the call is modified to 
\begin{lstlisting}
 tremolo_mpi project.tremolo
\end{lstlisting}

Both versions share the following parameters:
% The debug parametes will only be shown in the internal version.
\begin{lstlisting}
Usage: tremolo [-hvx ] [-a alarmtime] [-d debugstr|-D debugstr] [-n nicelevel] [-o verbosity] [-s sleeptime] [-f] mainparameterfile
  -a alarmtime  Sets alarm to alarmtime seconds. Code will finish after next timestep, writing out the current state.
  -f mainfile   Specify main parameter file (last argument: '-f' may be omitted)
  -h            Displays this help page and exits successfully.
  -n nicelevel  Decreases priority of process.
  -v            Increases verbosity level by one. Default value: 0.
  -V            Print version.
\end{lstlisting}

\draft{
Only for developers and in developer version the following options are accepted:
%\begin{lstlisting}
{\tt  -d debugstr   Starts debugger right away.

  -D debugstr   Starts debugger when encountering an error.
                Valid syntax for debugstr:
                0 sleep for a minute, no debugger
                [host[:display][,debugger]]  start debugger xterm on host:display
                (if DISPLAY is set or for local connection set host to local).
                Valid values for debugger are: gdb, dbx, ddd, cvd, totalview, ups.

  -s sleeptime  wait sleeptime seconds for debugger to come up (-d, -D)}
%\end{lstlisting}
\todo{Create new lst listing environment which is sensitive to the draft option}
}

\todo{detail verbosity syntax}

while the parallel version takes one additional parameter,
\begin{lstlisting}
  -p px,py,pz   Start parallel code with domain decomposition,
                using px slices on x-axis, py slices on y-axis and pz slices on z-axis (total: px*py*pz processes)
\end{lstlisting}
which must be supplied to actually benefit from parallel execution, since the default would be one process.
However, Tremolo-X will always check whether the number of processes in the MPI-call matches the number
of sub domains in the domain decomposition and will throw an error if this is not the case.


For parallel codes it may be necessary to specify the main parameter file
with an absolute path.

\draft{
\section{Using the GUI}

\todo{separate instructions for the GUI?}
}
