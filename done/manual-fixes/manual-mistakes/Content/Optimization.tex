
\chapter{Optimization}
\label{optimization}

While the optimization feature of Tremolo-X can be viewed as a non-dynamic
simulation in its own right, it is very beneficial to temper a simulations starting
ensemble prior to a dynamic simulation.

The optimization manifests itself in a minimization of the potential energy,
by rearranging particle positions and/or the simulation domain.

There are several methods available to minimize a given function value. Gradient based methods are computationally convenient for Molecular Dynamics simulations, since the
potential energy and its derivative need to be computed anyway. The method used by the Tremolo-X software is based on the modified \textbf{c}onjugated \textbf{g}radient (CG)
method by Polak-Ribi\'ere.\cite{Griebel.Hamaekers:2005b}

When comparing a \textbf{s}teepest \textbf{d}escent to a CG method, we notice that the SC is more stable than the CG if the starting position is distant from a local minimum point.
On the other hand the SD converges more slowly in the near neighborhood of such a point than the CG does. While it is possible to use both methods and switch after some iterations it is also possible to use a CG method with periodic resets to combine the benefits of both approaches.

Furthermore convergence can be stabilized by choosing \emph{(strong) Wolfe} conditions for the line search:

With $d_k$ the search direction from point $q_k$ (here denoting the vector of all particle positions) let:
\begin{align*}
\Phi(\eta) := E_{pot} \left( \textbf{q}_k + \eta \cdot \textbf{d}_k \right)
\Phi'(\eta) := \nabla E_{pot} \left( \textbf{q}_k + \eta \cdot \textbf{d}_k \right) \cdot \textbf{d}_k
\end{align*}
The so called \emph{(strong) Wolfe} condition is then the combination of the \emph{sufficient decrease} or \emph{Armijo} condition
\begin{equation*}
\Phi(\eta)\leq \Phi(0) + \alpha \eta \Phi'(0), \qquad 0 < \alpha < 1
\end{equation*}
and the \emph{curvature} condition
\begin{equation*}
\Phi'(\eta) \geq \beta \Phi'(0), \qquad 0 < \alpha < \beta < 1
\end{equation*}.

The \emph{Armijo} condition fixes an upper limit on acceptable new function values, whereas the \emph{curvature} condition imposes a lower bound on the step size $\eta$.

\todo{Choosing a suitable $\eta$ to satisfy the above conditions can be restricted to a certain interval (with respect to the original step length). Not implemented.}

Another method to control the steps of the iteration method is to apply a prefactor to the step, in essence reducing the maximum change of the ensemble\todo{without reset prefactor kills convergence?}.

The minimization iteration finishes, when the change in the mean force acting on a particle between consecutive iteration steps drops below a fixed threshold (drops below a fixed factor).

\bigbreak
The minimization may be extended to the structure of the simulation domain as well, by observing changes in the potential energy when scaling with the box matrix.
Depending on the type of box one wishes to retain from the input values, constraints have to be specified (to (de-)couple box edge length and or box angles).
The matrix takes inputs 0 and 1, specifying accessible degrees of freedom, in addition constraints have to be specified as off (\todo{is there even an off?}), 
none (no constraint, this is unphysical and should not be used), isotropic (box stays in cubic shape, only diagonal elements are changed[Andersen constraint]),standard (\todo{effect of standard}) and symmetric (\todo{effect of symmetric}).


Specifying an external pressure and/or stress values for the box is essential for the optimization results to conform with physical reality.


